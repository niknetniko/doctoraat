\documentclass[./main]{subfiles}
\usepackage{glossaries}

\begin{document}

This part of the manuscript deals with TESTed, an educational software test framework.
One of its defining features is the ability to create programming-language-agnostic exercises.
This means that the same exercise (with one test suite) can be solved in multiple programming languages, without loosing automated assessment.

%\marginnote{TESTed was first started in my master's thesis.}
TESTed has produced two publications:

\begin{itemize}
    \item \fullcite{strijbolTESTedEducationalTesting2023}
    \item TODO
\end{itemize}

The first paper is the answer to the research question: ``Is it possible to create an educational software test framework for programming-language-agnostic exercises?''.
The answer is yes.
\Cref{ch:tested1} is based on that article.
\marginnote{This manuscript doesn't have a page limit, after all.}
It has been expanded with more information about the inner workings of TESTed, in addition to going more in-depth on technical aspects of the framework.
Where necessary, the text has also been updated to reflect changes made to TESTed after publication.
However, one thing to keep in mind is that performance-related benchmarks have not been updated, as these are covered in the second chapter.

With a working prototype, we then took a step back to look at what is required to go from the prototype to a viable option for creating programming exercises.
We want TESTed to be suitable for both educators in higher education and secondary education.
Our ambition was to make TESTed the default options for creating programming exercises in Dodona~\autocite{vanpetegemDodonaLearnCode2023}, our platform for programming exercises.

This resulted in the second paper (\cref{ch:tested-dsl}).
We looked at software testing more broadly, and we conclude that an important part is an ergonomic and approachable way to author test suites for exercises.
The result of this is TESTed-DSL, a domain-specific language for authoring programming-language-agnostic exercises with support for automated assessment.
We also provide a solution to language-agnostic task descriptions.

As a software project, the source code for TESTed is important.
It is published under the MIT licence at \url{https://github.com/dodona-edu/universal-judge}.
This is the tradition at Team Dodona (of which TESTed is a part).

Two ``types'' of documentation are also available:

\begin{itemize}
    \item The guides, for educators wanting to create programming exercises: \url{https://docs.dodona.be/en/guides/exercises/}.
          Most of these guides are currently only available in Dutch.
    \item The reference documentation, for a more in-depth look or more technical subjects: \url{https://docs.dodona.be/en/references/tested/}.
          For example, this includes the documentation on how to extend TESTed to add support for more programming languages.
\end{itemize}

\end{document}
