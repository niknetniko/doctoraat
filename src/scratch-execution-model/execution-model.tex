\documentclass[../main]{subfiles}

\begin{document}

\chapter{The Scratch execution model}\label{ch:scratch-execution-model}

%\dictum[J. Maloney, M. Resnick, et al. \\ \textit{The Scratch Programming Language and Environment}]{Concurrency is often considered an advanced programming technique. Yet our everyday world is highly concurrent, so Scratch users
%are not surprised that a sprite can do several things at once.}

The programming language Scratch inherently supports parallel programming.
Each Scratch program (called a project) consists of a number of sprites with individual code (\cref{ch:scratch-the-programming-environment}).
Blocks that are attached together form scripts, and each sprite can have multiple scripts running concurrently as separate threads within the Scratch virtual machine.

The Scratch execution model combines a fixed-step time loop (30 frames per second) with an almost-cooperative threading model.
This means that threads are seldom interrupted, mostly relying on explicit yielding to other threads.
While this approach minimizes the occurrence of certain race conditions, some concurrency issues persist~\autocite{maloneyScratchProgrammingLanguage2010}.
For instance, the order in which sprites respond to broadcasts can be unpredictable.
Consequently, even without explicit concurrency controls, Scratch is a useful tool for teaching concurrency concepts~\autocite{fatourouTeachingConcurrentProgramming2018}.

However, the current execution model has some drawbacks.
The cooperative nature of the threading model can lead to unexpected behaviour when working with multiple sprites.
Additionally, the execution model complicates the use of debuggers within Scratch.
A traditional step function (which steps one block at a time) exposes execution states that are normally hidden.
Alternative step functions (see, for example, the one used by Blink in \cref{subsec:stepwise-execution}) diverge from the normal execution and are thus undesirable.

To address these issues, this chapter begins with an in-depth exploration of Scratch's current execution model.
This analysis is essential for understanding the subsequent section, where we explore the model's shortcomings in more detail.
We then propose some modifications to the execution model, which would solve the issues we have identified.
These modifications are finally evaluated to determine their impact on real-world Scratch projects in a preliminary benchmark.

\section{Elements of a Scratch program}\label{sec:elements-of-a-scratch-program}

\begin{listing}
    \centering
    \begin{sublisting}{0.45\textwidth}
        \centering
        \begin{scratch}[scale=0.7]
            \blockinit{when \greenflag{} clicked}
            \blockmove{move  \ovalnum{100} steps}
            \blockmove{turn \turnleft{} \ovalnum{90} degrees}
            \blockmove{move  \ovalnum{100} steps}
            \blockmove{turn \turnleft{} \ovalnum{90} degrees}
            \blockmove{move  \ovalnum{100} steps}
            \blockmove{turn \turnleft{} \ovalnum{90} degrees}
            \blockmove{move  \ovalnum{100} steps}
            \blockmove{turn \turnleft{} \ovalnum{90} degrees}
        \end{scratch}
    \end{sublisting}
    \begin{sublisting}{0.45\textwidth}
        \centering
        \begin{scratch}[scale=0.7]
            \blockinit{when \greenflag{} clicked}
            \blockrepeat{repeat \ovalnum{4}}{
                \blockmove{move  \ovalnum{100} steps}
                \blockmove{turn \turnleft{} \ovalnum{90} degrees}
            }
        \end{scratch}
    \end{sublisting}
    \caption{
        Two Scratch programs that seemingly exhibit the same behaviour: the sprite moves in a square of 100 steps, and finally stops at the same position as the start of the program.
    }
    \label{lst:scratch-two-programs}
\end{listing}

A Scratch program consists of zero or more sprites and a stage (see also \cref{ch:scratch-the-programming-environment}).
For every sprite at least one target is created (a target is what is drawn on the screen), while the stage has exactly one target.
All targets have their own local state: the variables and visual properties (e.g.\ position, size, bounding box, colour, direction).
Clones create more targets of the same sprite.
While clones have their own separate state, all targets based on the same sprite share the same code.
In the virtual machine, there is no substantial difference between how targets from different origins (sprites, stage, clones) are handled, so we can just consider targets for the remainder of this chapter.

Code-wise, the Scratch blocks are organized into categories (see \cref{subsec:using-the-environment-and-the-blocks}).
However, in this case, it is useful to look at their technical type, which corresponds to their shape.
In total, there are seven types of blocks:

\begin{enumerate}[noitemsep]
    \item Hat blocks \scratchinline{\blockinit{\hspace{1em}\dots\hspace*{1em}}}, which are placed at the start of a script (they are named hat blocks since they visually sit on top of a script).
        A script can only have one hat block.
        They function as event listeners, which trigger execution of the script if the event occurs.
    \item Stack blocks \scratchinline{\blockmove{\hspace{1em}\dots\hspace*{0.5cm}}}, representing program statements.
        These are the most common blocks.
        They are called stack blocks since they are stacked on top of each other.
        Stack blocks broadly fulfil the role of statements in Scratch.
    \item C blocks \scratchinline{\blockif{\hspace{1em}\dots\hspace*{1em}}{\blockspace[0.2]}}, which are named after their shape.
        They are used for most of the control flow blocks: loops and branches.
        The variant for the if/else block is sometimes called an E block since it has two slots.
    \item Reporter blocks \setscratch{baseline=c,scale=0.5}\ovalmove{\hspace{1em}\dots\hspace*{1em}}, act as variables or values and can be slotted into other blocks.
        Operators that result in a value also have this shape.
        The reporter blocks fulfil the role of expressions.
    \item Boolean blocks \setscratch{baseline=c,scale=0.5}\boolsensing{\hspace{1em}\dots\hspace*{1em}}, which are analogous to reporter blocks but result in a boolean.
    \item Cap blocks \scratchinline{\blockstop{\hspace{1em}\dots\hspace*{1em}}}, which end a script: no blocks can be added afterwards.
        Note that the infinite loop block, for example, is both a C block and a cap block.
    \item Custom blocks \scratchinline{\initmoreblocks{define\hspace{1em}\dots\hspace*{1em}}}, which define ``procedures''.
\end{enumerate}

\section{Related work}\label{sec:execution-related-work}

The Scratch execution model is defined by its implementation in the virtual machine.
There exists, at least to the knowledge of the authors, no comprehensive formal description of the execution model.
This does not mean there is no prior work.
From the Scratch team, \textcite{maloneyScratchProgrammingLanguage2010} provide a high-level description of the threading model.

Another body of works that provides insights into the Scratch execution model comes from the \emph{Chair of Software Engineering II} group, led by Gordon Fraser.
These publications all provide descriptions for parts of the execution model.

First, \textcite{stahlbauerTestingScratchPrograms2019} propose a formalization of three aspects in Scratch: the user perspective, a syntactic model and a semantic model.
They describe the semantics of Scratch with a memory model based on message passing.
Next, \textcite{stahlbauerVerifiedScratchProgram2020} develop LeILa, an intermediate language to which Scratch projects can be translated, with the intended use of performing analysis on Scratch projects.
The authors also provide a formalization of LeILa, using approximations for the behaviour of Scratch in some areas.
Also, \textcite{gotzModelbasedTestingScratch2022} model the state-based behaviour of Scratch programs using a finite state machine.
Finally, \textcite{deinerAutomatedTestGeneration2023} delve deeper into the actual execution of the virtual machine, while also proposing some modifications to it, for example, to make execution deterministic.

Other block-based languages also have to deal with concurrency.
For example, MakeCode also uses a non-preemptive threading model, inspired by Scratch~\autocite{ballMicrosoftMakeCodeEmbedded2019}.
There has also been some work on concurrency and concurrency controls in other block-based languages~\autocite{chungConCodeItComparisonConcurrency2020}.
However, since these languages do not use the Scratch virtual machine for execution, their relevancy for this chapter is limited.

\section{The current execution model}\label{sec:the-current-execution-model}

\subsection{Execution of a Scratch program}\label{subsec:execution-of-a-scratch-program}

\marginnote{Arguably, it is more of a concrete syntax tree, as e.g. the position of blocks is also saved. However, in Scratch's case, the differences are minimal, so we call it an abstract syntax tree, as Scratch themselves do.}
When executing Scratch code, the virtual machine transforms the blocks into an abstract syntax tree.
These are organized by target, and every execution of a script results in a distinct thread inside the virtual machine.
These are green threads: implemented fully in the virtual machine.

The virtual machine is thus responsible for scheduling these threads.
%\Cref{fig:blink-architecture} gives a schematic overview of the interaction between the different parts.
It uses an almost-cooperative threading model, which \textcite{maloneyScratchProgrammingLanguage2010} call the ``Scratch threading model''.
This means it is mostly non-preemptive: the virtual machine will not interrupt threads at arbitrary points in their execution.
The threads must voluntarily yield control or reach a limited set of points in their execution.
The rational is given in~\cite{maloneyScratchProgrammingLanguage2010}: ``Scratch builds concurrency control into its threading model in a way that avoids most race conditions, so that users do not need to think about these issues.
This is done by constraining where thread switches can occur.''.

At four well-defined points, a thread always yields, thus causing said thread switching:
\begin{enumerate}
    \item When a block with a fixed duration is executed.
        There are a number of blocks that fall under this category.
        \scratchinline{\blockcontrol{wait \ovalnum{}}} is an obvious inclusion, but this also applies to \scratchinline{\blockmove{glide \ovalnum{} secs to x: \ovalnum{} y: \ovalnum{}}}, for example.
        \scratchinline{\blocksound{play sound \ovalsound*{} until done}} also falls under this category, even if there is no explicit time.
    \item When a block waits on execution of other blocks.
        For example, \scratchinline{\blockevent{broadcast \selectmenu{something} and wait}}.
    \item The last block of a loop (thus \scratchinline{\blockinfloop{forever}{\blockspace[0.2]}}, \scratchinline{\blockrepeat{repeat \ovalnum{}}{\blockspace[0.2]}}, and \scratchinline{\blockrepeat{repeat until \boolempty[1em]{}}{\blockspace[0.2]}}).
        This means thread switching will occur after every loop iteration.
    \item A recursive procedure call is detected.
        Scratch attempts to detect these (up to five levels of indirection) and will yield the thread on each call if it detects a recursive call.
\end{enumerate}

There is one exception: when using procedures ``without screen refresh'', Scratch will interrupt a thread that runs longer than \qty{500}{\milli\second}.
This is called the ``wrap timer'', and has some curious edge cases.\footnote{\url{https://github.com/scratchfoundation/scratch-vm/issues/2834}}

The threads are executed in a first-come, first-serve manner: there are no priorities nor changes in thread order.
The first thread is executed until it yields or ends, then the next thread, and so on.
We call the execution within one thread until it yields or ends a \term{turn}.
A thread can have one of three conceptual states: \emph{done}, \emph{running}, and \emph{yield}.

\begin{figure}
    \centering
    \includestandalone{scratch-model}
    \caption{Overview of the interplay between the threading model and the ``game loop''. Within one step (which is done 30 times per second), one or more ticks are executed. The arrow with \CircledText{2} illustrates this: after the first tick, another is started if less than \qty{75}{\percent} of the step time (the time one step has to complete, \qty{33}{\milli\second}) has been used, and a redraw has not been requested, and Scratch is not in turbo mode. Within one tick, a turn is executed for each thread \CircledText{1}: a thread executes until it terminates or the thread yields.}
    \label{fig:scratch-model-explained}
\end{figure}

\marginnote{Scratch 3 should actually run at \qty{60}{\fps}; however Scratch enables a compatibility mode by default, resulting in \qty{30}{\fps}.}
The virtual machine uses a \emph{fixed-time step with synchronization} main loop~\autocite{nystromGameProgrammingPatterns2014}, also called a \emph{synchronized coupled model}~\autocite{valenteRealTimeGame2005}.
This means that the virtual machine runs in \term{steps}: internally, the \texttt{step} function is called every \qty{33}{\milli\second} (so 30 times a second, commonly known as \qty{30}{\fps}).

In each step, the virtual machine will execute one or more ticks.
A \term{tick} is one turn in every thread: the first thread is executed until it yields or terminates, then the second thread and so on.
After the tick, a redraw is performed if needed (in practice this is always done, as the source code contains a to-do to implement selective redrawing).
After the first tick is finished, the virtual machine decides whether to run another tick (\cref{fig:scratch-model-explained}).
A new tick is started if less than \qty{75}{\percent} of the step time (the time one step has to complete, \qty{33}{\milli\second}) has been used and a redraw has not been requested.
Note that once a tick has started, it is run completely and cannot be stopped.
The arbitrary \qty{75}{\percent} is intended to prevent frame drops: steps that take longer than their allocated step time, meaning the next step is delayed.
Also, in practice, many blocks request a redraw, so in many Scratch projects, a step only ever runs one tick.

\begin{figure}
    \centering
    \begin{subfigure}{0.40\textwidth}
        \includestandalone{threading-no-loop}
    \end{subfigure}
    \begin{subfigure}{0.59\textwidth}
        \includestandalone{threading-with-loop}
    \end{subfigure}
    \caption{The execution of the two programs from \cref{lst:scratch-two-programs}. In the unrolled version (left), all code is executed in the first turn, meaning only one tick and step is needed. In the version with loop (right), the loop yields after each iteration, meaning the rest of the step is filled with idle time. In total, four steps are needed. }
    \label{fig:scratch-two-execution}
\end{figure}

A concrete example of the execution model is given by \cref{fig:scratch-two-execution}, which shows the execution of the programs from \cref{lst:scratch-two-programs}.
These programs have only one thread.
Since none of the blocks in the unrolled program yield the thread, the full program is executed in one tick.
In the other version, with a loop, the thread yields after each iteration of the loop, meaning the program needs four steps.
This does result in an observable difference: in the unrolled program, the sprite does not move visually.
As a redraw only happens between steps, the sprite is back at its original position.
In the looped version, the sprite moves four times (albeit rapidly), as there is a redraw between each step.

\subsection{Implementation details}\label{subsec:implementation-details}

How different parts of the virtual machine implement the execution model from \cref{subsec:execution-of-a-scratch-program} is shown in \cref{fig:blink-architecture}.
When the user interface loads a project, it also starts the virtual machine.
This means that the game loop is active (this is done in the class \texttt{Runtime}).

New threads are only created in two scenarios:
\begin{itemize}[noitemsep]
    \item Code needs to be executed, either because an event triggered some hat blocks (green flag, key press, etc.) or because the user clicked on some blocks.
    \item When ``stage monitors'' or watchers are active.
        These are used in the user interface to show the value of variables or properties.
        The watchers for variables also allow the user to change the value of the variable.
\end{itemize}

The \texttt{Runtime} calls the method \mintinline{javascript}{stepThreads} in the \texttt{Sequencer} class.
This class is responsible for implementing the ticks.
After each tick, done threads are removed, and a new tick is started if possible.
It is also here that the thread status is managed.
While there are three conceptual statuses, the implementation has five:
\begin{description}[noitemsep]
    \item[\texttt{done}] The thread has finished executing all blocks and will be removed after this tick.
    \item[\texttt{running}] The thread is being executed and has more blocks to execute.
        It will be scheduled again next tick.
    \item[\texttt{yield}] The thread is waiting an amount of time.
        The thread is scheduled again next tick to see if the wait time is over.
    \item[\texttt{promise wait}] The thread is waiting for a JavaScript promise to be resolved, after which the thread will be set to \texttt{running}.
    \item[\texttt{yield tick}] The thread yields until the next step.
        The purpose of this status seems to be some performance optimizations to aid with benchmarking.\footnote{\url{https://github.com/scratchfoundation/scratch-vm/pull/1211}}
\end{description}

Each thread maintains a stack structure.
The \texttt{Sequencer} will then look up the next block on said stack, and if there is one, it will call the \texttt{Execute} class.
That class will actually execute the block on the stack.
For normal blocks (somewhat confusingly called ``stack blocks'', since they stack together to form a script), the current block is popped from the stack, the block is executed, and the next block is put on the stack.
The stack is only useful when working with C-blocks or procedures.
For example, C blocks will push the first block in their slot on the stack.
In the case of a loop, a counter is saved in the stack frame to determine how many times the loop should be run.
The exact implementation of the stack is less relevant for this chapter, so it is left to the reader to browser the source code.

\section{Limitations of the execution model}\label{sec:limitations-of-the-execution-model-for-the-debugger}

This section illustrates a few limitations of the current execution model, first in general and then specifically for a debugger.

\subsection{During general execution}\label{subsec:in-general}

\begin{listing}
    \centering
    \begin{scratch}[scale=0.6]
        \blockinit{when \greenflag{} clicked}
        \blockmove{go to x: \ovalnum{0} y: \ovalnum{112}}
        \blockpen{erase all}
        \blockpen{set pen color to \pencolor{ppcolor}}
        \blockpen{set pen size to \ovalnum{5}}
        \blockpen{pen down}
        \blockmove{point in direction \ovalnum{108}}
        \blockrepeat{repeat \ovalnum{4}}{
            \blockmove{move \ovalnum{80} steps}
            \blockmove{turn \turnright{} \ovalnum{36} degrees}
            \blockmove{move \ovalnum{80} steps}
            \blockmove{turn \turnright{} \ovalnum{36} degrees}
        }
        \blockpen{pen up}
    \end{scratch}
    \hspace{3em}
    \begin{scratch}[scale=0.6]
        \blockinit{when \greenflag{} clicked}
        \blockinfloop{forever}{
            \blockif{if \boolsensing{touching color \pencolor{blue}} then}{
                \blockstop{stop \selectmenu{all}}
            }
        }
    \end{scratch}
    \caption{The implementation, with a bug in the first script (left) and a non-working second script (right).}
    \label{lst:star-model-implementation}
\end{listing}

\begin{figure}
    \begin{subfigure}{0.45\textwidth}
        \includegraphics[width=\textwidth]{star-before}
        \caption{The stage before execution.}
        \label{fig:star-exercise-model-before}
    \end{subfigure}
    \hfill
    \begin{subfigure}{0.45\textwidth}
        \includegraphics[width=\textwidth]{star-after}
        \caption{The stage after execution.}
        \label{fig:star-exercise-model-after}
    \end{subfigure}
    \caption{Result of running the implementation from \cref{lst:star-model-implementation} for the \emph{Star} exercise.}
    \label{fig:star-exercise-model}
\end{figure}

As \textcite{maloneyScratchProgrammingLanguage2010} mentioned, the Scratch threading model does not solve all issues with concurrency.
To illustrate this point, we will consider a variant on the \emph{Star} exercise from \cref{subsec:star-exercise}.
In this exercise, the goal is to let a sprite move around on a path without falling into the water (read: touching a blue colour).
\Cref{lst:star-model-implementation} shows an implementation for this exercise with an additional script that will stop execution if the sprite touches something blue.
We asked a number of educators that had experience with Scratch to predict the behaviour of this implementation.
All of them expected the execution to stop either when the sprite first touches the water or after executing the block when the sprite first touches the water.
However, as can be seen in \cref{fig:star-exercise-model}, where the canvas is shown before and after running the code, the second script that should have stopped execution did not work.

The reason for this is the non-preemptive thread switching: the body of the loop is always executed atomically.
At the start of the loop, the sprite does not touch the water.
After execution of one iteration, the sprite is back on the path and does not touch the water.
Therefore, whenever the second script is executed, the sprite is not touching the water, which explains why the execution was not stopped.

Consider the original code in the \emph{Star} exercise from \cref{subsec:star-exercise}.
The second script there does not stop the execution, but uses Blink's pause block to halt the execution.
Using the current execution model, the pause block will not function for the same reasons mentioned above.

\subsection{Specifically for a debugger}\label{subsec:specifically-for-a-debugger}

A fundamental feature of any debugger is the ability to step through code: executing one statement and then pausing the execution to facilitate inspection of the program state.
The functionality is also essential in debuggers for Scratch: all existing debuggers for Scratch implement it.
In Scratch, executing a single statement translates to executing a single block.

However, traditional single-block stepping has some drawbacks in Scratch.
A first drawback is that users have to click a lot, since the stepping functionality is global, not per thread.
Secondly, and more importantly, the step functionality exposes details of the Scratch execution model to the users.
For example, thread switching, which is normally implicit in Scratch's perceived parallel execution, becomes visible to the users during debugging.

Additionally, debuggers must choose what to do with intermediate states that are typically hidden during normal execution.
For instance, five consecutive blocks would result in one redraw in the normal execution.
One choice is to not alter the redraw logic (thus only redrawing when the normal execution would redraw), but this results in steps having no visual impact, even if the block logically should change the visuals.
The other choice is to redraw after every step.
While the effect of every block (and step) is then visible, this exposes intermediary steps that would normally not be drawn.
Some blocks (for example, checking if a sprite touches a colour) use the visual state, meaning these additional redraws can result in a different execution of the project.

As described in \cref{sec:blink-software-architecture}, our debugger Blink takes a different approach to the stepping feature.
We believe it is useful to maintain the observed parallelism of Scratch in the debugger: we define a step in the debugger as executing one block in every thread.
This approach allows users to keep focus on relevant threads without distractions from thread switches, which can be cumbersome in complex programs.
Users can thus focus on the script(s) they believe are involved in the failure while ignoring (correct) scripts running at the same time.

This approach does come at a price: it changes the Scratch execution model, which is not trivial due to two main considerations:

\begin{enumerate}
    \item If the execution model is only changed when debugging, the debugger does not debug the same program execution as when running the program.
        This can result in different behaviour, meaning the bugs for which the debugger is used need no longer be present or new bugs, unique to the debugger, could be introduced.
    \item If the execution model is changed, we need to ensure that existing Scratch programs keep working and that we do not introduce concurrency problems, as the current execution model of Scratch was explicitly chosen to avoid those.
\end{enumerate}

We opt for the second option: modifying the Scratch execution model.
In the next section, we discuss what we changed, after which we investigate the impact on performance and behaviour of existing Scratch projects.

\section{Towards a new execution model}\label{sec:a-family-of-new-execution-models}

In line with how we want the stepping feature of the debugger to work, we have decided to change the Scratch execution model as follows: we modify a turn to execute exactly one block before yielding.
Thus, in a single tick, the virtual machine will execute a single block in every thread.

As there is often only one tick per step (due to many blocks requesting redraws), this means that only one block would be executed per step (thus one block for every thread per \qty{33}{\milli\second}).
Consequently, this makes execution slower than in the original execution model.

A possible remedy is to modify the number of steps that are taken.
For example, it might be better to run at twice or more times the number of steps per second.
In turbo mode, this would mean the steps are done as fast as the hardware allows.
While this does make everything go much faster, it does introduce a big difference in execution time between a performant machine and a slower machine.

These changes to the execution model have as a benefit that the \emph{Star} solution (\cref{lst:star-model-implementation}) will behave as expected, since the first thread will yield after the first block in the loop.
It can also illustrate that this does introduce concurrency considerations that were not present in the original execution model.
For example, if the conditional block in the second thread evaluates to true, the first thread will execute another block before execution is stopped by the block inside the conditional block from thread two.

In the next section, we analyse existing Scratch projects to determine which frame rate for the new execution model most closely results in the same execution speed as the original execution model (which runs at 30 frames per second).
We also take this opportunity to analyse the complexity and block use in Scratch projects, to evaluate whether the concurrency considerations would cause problems.

\section{Exploration of Scratch projects}\label{sec:evaluation-of-scratch-projects}

As Scratch is used by many people, it is important that changes to the execution model do not adversely affect existing Scratch projects.
However, this requires us knowing what Scratch projects look like.
The aim of this analysis is to determine what blocks are used in Scratch projects, how big projects are, and what programming concepts are used.

We begin by looking at existing work on analysing Scratch projects, followed by our own analysis.

\subsection{Existing analyses}\label{subsec:existing-analyses}

\makenote*{There is some discussion if the cyclomatic complexity is a useful metric. It might have no more predictive ability than lines of code \autocite{hattonInvitedTalkRole2008,fentonCritiqueSoftwareDefect1999,cherfInvestigationMaintenanceSupport1992}.\vspace{0.5cm}}
\Textcite{aivaloglouHowKidsCode2016} analysed \num{250000} Scratch 2.0 projects they scraped from the public Scratch site.
They looked at the types of blocks used, the size of the projects, and the complexity.
For the complexity, they utilize the cyclomatic complexity metric~\autocite{mccabeComplexityMeasure1976}.
The considered decision points are the \texttt{if} and \texttt{if-else} blocks.

They found that most Scratch projects are small: \qty{75}{\percent} have less than 5 sprites, 12 scripts, and 76 blocks.
\qty{25}{\percent} has less than 12 blocks, although there are some huge projects with more than \num{20000} blocks.
They also found that about \qty{78}{\percent} of projects have no decision points.

\Textcite{fronzaApproachEvaluateComplexity2020} investigate Scratch projects with different complexity metrics.
The dataset is, however, much more limited: 80 projects were analysed.
\makenote*{The usefulness of the Halstead metrics is even more controversial~\autocite{hamerHalsteadSoftwareScience1982,shenSoftwareScienceRevisited1983,jonesDimensionalAnalysisHalstead2019}.}
The authors also measure the cyclomatic complexity, in addition to some Halstead complexity measures~\autocite{halsteadElementsSoftwareScience1977}, and their own proposal for a ``when'' metric.
They do use more decision points for the cyclomatic complexity (\texttt{if}, \texttt{if-else}, \texttt{repeat until}, \texttt{wait until}, \texttt{and}, \texttt{or}).
The proposed ``when'' metric counts the number of ``when'' blocks (e.g.\ hat blocks with certain conditions).

\subsection{A new dataset of Scratch 3.0 projects}\label{subsec:a-new-dataset-of-scratch-projects}

Since the dataset used by \textcite{aivaloglouHowKidsCode2016} consists of Scratch 2.0 projects and \textcite{fronzaApproachEvaluateComplexity2020} only analyse 80 projects, we found it necessary to collect a new dataset of Scratch projects.

We constructed a new dataset as follows, using the Scratch website.\footnote{\url{https://en.scratch-wiki.info/wiki/Scratch_API}}
Creating a new project provides an identifier ({\addfontfeature{Numbers={Lining,Proportional}}996725074}, April 7th, 2024), which we used as a starting point.
We then subtract one from the identifier, downloaded the project if possible, and continued.
\marginnote{The \num{1129465} projects were created in approximately 3 days, illustrating Scratch's popularity.}
The oldest project in the dataset is from April 5th, 2024, with identifier {\addfontfeature{Numbers={Lining,Proportional}}995595608}.

This resulted in \num{237926} downloaded public projects (of the total \num{1129465} that were made between our newest and oldest projects).
From those, \num{207} were corrupt or for an older version of Scratch.
We also filtered out the following projects: \num{37936} (\qty{15.9}{\percent}) were empty and \num{4411} (\qty{1.9}{\percent}) had no executable code (e.g.\ only head blocks or scripts without head blocks).
This results in a final dataset of \num{195372} Scratch projects we considered for further analysis.

\subsection{Analysing Scratch 3.0 projects}\label{subsec:analysing-scratch-3.0-projects}

Hairball~\autocite{boeHairballLintinspiredStatic2013} was commonly used to analyse Scratch projects (it is also used by the existing analyses), but does not support Scratch 3.0.
For this reason, we have implemented a similar tool in JavaScript (versus Hairball's use of Python).\footnote{\url{https://github.com/scratch-ed/scratch-analysis}}
It also supports plugins to support extensions for other analyses.
Being written in JavaScript, it has the advantage that it can reuse parts of the Scratch virtual machine, like reading and parsing Scratch projects.

Wherever possible, we have used the same definitions and metrics as used by \textcite{aivaloglouHowKidsCode2016}, for ease of comparison.
Whenever a direct comparison is possible and relevant, we have included their data in \textit{italics}.
For example, \qty{75}{\percent} (\textit{\qty{60}{\percent}}) indicates our data shows \qty{75}{\percent}, while \citeauthor{aivaloglouHowKidsCode2016} found \qty{60}{\percent}.

\subsection{Use of blocks}\label{subsec:use-of-blocks}

Scratch blocks can be categorized into seven types, based on their shape (\cref{sec:elements-of-a-scratch-program}), their usage shown in \cref{fig:block-shapes}.
Blocks can also be put into categories (\cref{subsec:using-the-environment-and-the-blocks}).
The number of projects that use a block from a certain category is shown in \cref{fig:block-categories}.
Note that these numbers do not fully compare with \textcite{aivaloglouHowKidsCode2016}: since then, some new Scratch extensions were added, and the pen-related blocks have moved to an extension.

It is notable that extensions are not widely used: only \qty{12.2}{\percent} of projects use any extension.
The Pen extension is the most popular one, appearing in \qty{6.9}{\percent} of the projects.

\begin{figure}
    \centering
    \includestandalone{block-shapes}
    \caption{
        Number of blocks by shape in all projects.
        The total number of blocks is \num{47808628}.
        Note that the \textit{forever} block is counted twice (as a cap block and a C block), and procedure-defining blocks are counted as hat blocks.
    }
    \label{fig:block-shapes}
\end{figure}

\begin{figure}
    \centering
    \includestandalone{block-categories}
    \caption{
        Number of projects that use blocks from a certain category.
        Custom blocks are excluded, and all blocks for extensions are counted together.
        Bar colours correspond to the block category colours in the traditional Scratch 3.0 colour scheme.
    }
    \label{fig:block-categories}
\end{figure}

\subsection{Size and complexity}\label{subsec:size-and-complexity}

\begin{table}
    \centering
    \caption{
        Size and complexity statistics about the \num{195372} non-empty Scratch projects in our dataset.
        Unless otherwise noted, all numbers are shown per project and blocks are counted as logical lines of code.
        The first column of numbers reports the mean from \textcite{aivaloglouHowKidsCode2016} if available.
        The subsequent numbers are, in order, the mean and the five-number summary: the minimum, the first quartile, the second quartile (the median), the third quartile, and the maximum.
    }
    \label{tab:loc-scratch}
    \begin{wide}
        \addfontfeatures{Numbers={Monospaced,Lining}}
        \begin{tabular}{|l|S S|S[table-format=2.0]|S[table-format=2.0]|S[table-format=2.0]|S[table-format=2.0]|S[table-format=5.0]|}
            \hline
            {} & {\AtNextCite{\defcounter{maxnames}{1}}\citeauthor{aivaloglouHowKidsCode2016}} & {mean} & {min} & {Q\textsubscript{1}} & {Q\textsubscript{2}} & {Q\textsubscript{3}} & {max} \\
            \hline
            sprites (with code) & 5.68 & 4.98 & 1 & 1 & 2 & 5 & 1000 \\
            scripts (with code) & 17.35 & 19.73 & 1 & 2 & 4 & 11 & 9134 \\
            \hline
            blocks (logical lines) & 154.55 & 203.14 & 2 & 9 & 22 & 75 & 24084 \\
            blocks (physical lines) & {\textsc{n/a}} & 150.81 & 2 & 8 & 20 & 65 & 20249 \\
            dead blocks & {\textsc{n/a}} & 42.25 & 0 & 0 & 0 & 3 & 14912 \\
            \hline
            blocks per script & {\textsc{n/a}} & 10.30 & 1 & 2 & 5 & 10 & 5497 \\
            \hline
            \shortstack[l]{cyclomatic complexity \\ per script} & 1.58 & 1.85 & 1 & 1 & 1 & 2 & 5497 \\
            \hline
        \end{tabular}
    \end{wide}
\end{table}

\Cref{tab:loc-scratch} shows a summary of the project size for our dataset.
In the rest of this subsection, we detail some choices we made in analysing the projects and draw some conclusions.

When considering the size of a program, a frequently used metric is lines of code.
However, there is no universal agreed-upon manner in which to count lines of code~\autocite{nguyenSLOCCountingStandard2007}.
Two variants are frequently used: physical lines of code (the number of lines in the source files) and logical lines of code (an approximation of the number of statements or expressions).
While normally counting the physical lines of code is easy and counting logical lines requires some consideration, the reverse is true in Scratch.
To count logical lines of code, we can simply count all blocks.
For physical lines of code, we chose to count the number of ``main'' blocks in a script.
This means we do not count blocks used as arguments, e.g.\ the condition block of a loop is not counted.

When counting the number of scripts per project, we excluded scripts that consist only of a hat block, as these do nothing.
We similarly excluded sprites without code from the count of sprites per project.
These can have a role in some cases but are not useful in the statistics.

From these data, we can see that \qty{75}{\percent} of projects have less than 5 sprites, 11 scripts, and 80 blocks (\textit{\qty{75}{\percent} of projects have less than 5 sprites, 12 scripts, and 76 blocks}).
We can conclude that most Scratch projects are still small in Scratch 3.0.

\begin{figure}
    \begin{wide}
        \includestandalone[width=\linewidth]{cyclomatic-complexity}
    \end{wide}
    \caption{
        Distribution of scripts based on their cyclomatic complexity.
        Scripts with a complexity higher than 10 have been bundled into the last bucket.
    }
    \label{fig:scratch-cc}
\end{figure}

For the cyclomatic complexity, we used the same decision points as \citeauthor{aivaloglouHowKidsCode2016}.
\Cref{fig:scratch-cc} shows the distribution of the cyclomatic complexity in the Scratch projects.
Most scripts (\qty{72.6}{\percent}, \textit{\qty{78}{\percent}}) do not contain any decision points.
Additionally, another \qty{15.3}{\percent} (\textit{\qty{13.8}{\percent}}) has just one decision point.
On the other end of the spectrum, \qty{1.6}{\percent} has a complexity larger than 10, and 667 scripts (\qty{0.00017}{\percent}, \textit{\qty{0.000052}{\percent}}) have a complexity larger than 100.

We can conclude that most projects are small, since \qty{75}{\percent} has less than 5 sprites, 11 scripts, and 80 blocks.
Most scripts (\qty{72.6}{\percent}) have no decision points.

\subsection{Programming concepts}\label{subsec:programming-concepts}

\begin{table}
    \centering
    \caption{
        Prevalence of programming concepts in Scratch projects.
        The third column shows the results found by \textcite{aivaloglouHowKidsCode2016} in percentage if available.
    }
    \label{tab:scratch-programming-concepts}
    \begin{tabular}{|l|S[table-format=6.0] S|S|}
        \hline
        {Concept} & {№ of projects} & \% & {\AtNextCite{\defcounter{maxnames}{1}}\citeauthor{aivaloglouHowKidsCode2016} (\%)} \\
        \hline
        User input blocks & 90703 & 46.43 & 56.24  \\
        Random & 67403 & 34.50 & {\textsc{n/a}} \\
        \hline
        Conditional statements & 80664 & 41.29 & 39.81 \\
        Loop statements & 151050 & 77.31 & 77.18 \\
        Repeat with condition & 28526 & 14.60 & 13.59 \\
        \hline
        Variables & 67526 & 34.56 & 31.51 \\
        Lists & 18331 & 9.38 & 4.01 \\
        \hline
        Procedures & 33319 & 17.05 & 7.70 \\
        \hline
    \end{tabular}
\end{table}

\Cref{tab:scratch-programming-concepts} is an overview of the prevalence of some programming concepts in the analysed Scratch projects.
While the number of projects that use procedures (\qty{17.1}{\percent}, \textit{\qty{7.70}{\percent}}) is higher, it is still not used that much: a majority of projects do not use it.
Most projects do use loop statements (\qty{77.3}{\percent}, \textit{\qty{77.18}{\percent}}), yet the number of projects using a conditional loop is much smaller (\qty{14.6}{\percent}, \textit{\qty{13.59}{\percent}}).
Less than half of projects use conditional statements (\qty{41.29}{\percent}, \textit{\qty{39.81}{\percent}}), and a bit less than half (\qty{46.43}{\percent}, \textit{\qty{56.24}{\percent}}) use user input blocks.
About a third (\qty{34.56}{\percent}, \textit{\qty{31.51}{\percent}}) of the projects use variables, and only \qty{9.38}{\percent} (\textit{\qty{4.01}{\percent}}) uses lists.
This means that a large number of projects is simple (and this is what we would expect, given the previous metrics on project size and complexity).

In summary, most Scratch projects are small and simple.
However, big and complex ones do exist.
Scratch projects have not changed significantly since the analysis by \textcite{aivaloglouHowKidsCode2016}, even if Scratch 3.0 was released in that period.
However, this was expected: Scratch 3.0 did not introduce major changes to Scratch-the-programming-language.
The main differences are as follows: the number of complex projects and of projects with procedures increased slightly, while user input blocks are used a bit less.

\section{Evaluation of the new execution model}\label{sec:evaluation-of-the-new-execution-model}

To ascertain the effect of the new execution model on existing Scratch projects, we perform and report on a preliminary benchmark.
We measure the performance and behaviour of various projects using the existing Scratch 3.0 execution model and variations of the new execution model.

We consider five variations of the new execution model, which differ in how fast they run: \textbf{\textsc{em-30}} runs at \qty{30}{\fps}, \textbf{\textsc{em-60}} at \qty{60}{\fps}, \textbf{\textsc{em-90}} at \qty{90}{\fps}, \textbf{\textsc{em-120}} at \qty{120}{\fps}, and \textbf{\textsc{em-asap}} runs as fast as possible, meaning a new frame is started as soon as the previous frame finishes.

\subsection{Selection of projects}\label{subsec:selection-of-projects}

First, we differentiate between large and small projects.
In the previous analysis, we determined that \qty{75}{\percent} of projects have less than 5 sprites, 11 scripts, and 80 blocks.
We thus consider projects small if they have less than 80 blocks.

Secondly, we differentiate between projects with user interaction and those without user interaction.
While \qty{46.43}{\percent} of projects require user interaction, this makes those projects much more difficult to automatically benchmark.
We therefore only manually look at two such projects.

\subsection{Non-interactive projects}\label{subsec:non-interactive-projects}

\begin{figure}
    \begin{wide}
        \begin{subfigure}{\linewidth}
            \centering
            \includestandalone{block-comparison-legend}
        \end{subfigure}
        \begin{subfigure}{0.49\linewidth}
            \includestandalone{block-comparison-small}
            \caption{Small projects}
            \label{fig:blocks-non-interactive-small}
        \end{subfigure}
        \begin{subfigure}{0.49\linewidth}
            \includestandalone{block-comparison-large}
            \caption{Large projects}
            \label{fig:blocks-non-interactive-large}
        \end{subfigure}
        \par\bigskip
        \begin{subfigure}{0.49\linewidth}
            \includestandalone{block-violin-small}
            \caption{Small projects}
            \label{fig:blocks-non-interactive-violin-small}
        \end{subfigure}
        \begin{subfigure}{0.49\linewidth}
            \includestandalone{block-violin-large}
            \caption{Large projects}
            \label{fig:blocks-non-interactive-violin-large}
        \end{subfigure}
    \end{wide}
    \caption{
        Variants of the new execution model compared against the original Scratch 3.0 execution model.
        The top figures show the number of projects that execute more, equal, or fewer blocks than the original execution model.
        The bottom figures show the difference in the number of block executions compared to the original execution model.
        A negative number indicates that fewer blocks were executed than in the original execution model.
    }
    \label{fig:blocks-non-interactive}
\end{figure}

The benchmark dataset for non-interactive projects consists of 100 randomly chosen projects from the analysis dataset.
Of those 100 projects, 75 are small projects.
For these projects, we measure the number of executed blocks.
This counts the number of times every block (from the physical lines of code, so excluding arguments) is executed.
The benchmark dataset contains both projects that end and those that do not.
Projects that do not end are, for example, those with repeat forever blocks.
For projects that end, we measure the total number of executed blocks.
Non-ending projects are halted after \qty{60}{\second}, and the number of executed blocks within that time is counted.

\Cref{fig:blocks-non-interactive-small,fig:blocks-non-interactive-large} show how many projects differ in behaviour for small and large projects respectively.
\Cref{fig:blocks-non-interactive-violin-small,fig:blocks-non-interactive-violin-large} then show how much the projects differ.
The results show that most projects do not differ a lot, but there are a few outliers, particularly with the \textsc{em-asap} model, and with larger projects.
In general, the \textsc{em-90} or \textsc{em-120} model seems to provide the most similar behaviour for most projects.

However, the behaviour of the \textsc{em-asap} model warrants future research.
Our initial investigation of those outlier projects did not reveal an immediate cause why the behaviour is so different.
One hypothesis is that this is due to how the models are implemented: there are some quirks with JavaScript's \texttt{setTimeout} function, which is what is used in the virtual machine.
An alternative implementation of the \textsc{em-asap} model might be warranted.

\subsection{Interactive projects}\label{subsec:interactive-projects}

\begin{figure}
    \begin{wide}
        \begin{subfigure}[T]{0.49\linewidth}
            \includegraphics[width=\linewidth]{lightning-game}
            \caption{
                The \textit{Lightning} project in action.
                The user must use the arrow keys to move the cat around, avoiding the lightning bolts.
            }
            \label{fig:the-game-lightning}
        \end{subfigure}
        \begin{subfigure}[T]{0.49\linewidth}
            \includegraphics[width=\linewidth]{apple-game}
            \caption{
                The \textit{Catch the apples} project in action.
                The basket follows the mouse, and the user must catch as many apples as possible, without any apples falling on the ground.
            }
            \label{fig:the-game-apple}
        \end{subfigure}
    \end{wide}
    \caption{
        Overview of the two games we discuss.
    }
    \label{fig:the-game}
\end{figure}

\minisec{\textit{Lightning} project}

\begin{figure}
    \begin{subfigure}{\linewidth}
        \centering
        \includestandalone{lightning-scatter-original}
        \caption{
            The original implementation (see bottom left).
            The \textsc{em-30},\textsc{em-60}, and \textsc{em-90} did not achieve 10 points.
        }
        \label{fig:lightning-behaviour-original}
    \end{subfigure}
    \par\medskip
    \begin{subfigure}{\linewidth}
        \centering
        \includestandalone{lightning-scatter-procedure}
        \caption{A modified implementation (see bottom right)}
        \label{fig:lightning-behaviour-procedure}
    \end{subfigure}
    \par\medskip
    \begin{subfigure}{0.49\linewidth}
        \centering
        \begin{scratch}[scale=0.5]
            \blockinit{when \greenflag clicked}
            \blockinfloop{forever}{
                \blockif{if \boolsensing{key \selectmenu{right arrow} pressed?} then}{
                    \blockmove{change x by \ovalnum{10}}
                }
                \blockif{if \boolsensing{key \selectmenu{left arrow} pressed?} then}{
                    \blockmove{change x by \ovalnum{-10}}
                }
                \blockif{if \boolsensing{key \selectmenu{up arrow} pressed?} then}{
                    \blockmove{change y by \ovalnum{10}}
                }
                \blockif{if \boolsensing{key \selectmenu{down arrow} pressed?} then}{
                    \blockmove{change y by \ovalnum{-10}}
                }
            }
        \end{scratch}
        \caption{
            Original implementation of the interactive component.
        }
        \label{fig:the-game-lightning-code-original}
    \end{subfigure}
    \begin{subfigure}{0.49\linewidth}
        \centering
        \begin{scratch}[scale=0.5]
            \blockinit{when \greenflag clicked}
            \blockinfloop{forever}{
                \blockmoreblocks{move}
            }
        \end{scratch}
        \begin{scratch}[scale=0.5]
            \initmoreblocks{define \namemoreblocks{move}}
            \blockif{if \boolsensing{\dots} then}{
                \blockmove{change x by \ovalnum{10}}
            }
            \blockif{if \boolsensing{\dots} then}{
                \blockmove{change x by \ovalnum{-10}}
            }
            \blockif{if \boolsensing{\dots} then}{
                \blockmove{change y by \ovalnum{10}}
            }
            \blockif{if \boolsensing{\dots} then}{
                \blockmove{change y by \ovalnum{-10}}
            }
        \end{scratch}
        \caption{
            Alternative implementation of the interactive component.
            The four if blocks have been moved to a procedure ``run without screen refresh''.
        }
        \label{fig:the-game-lightning-code-better}
    \end{subfigure}
    \caption{
        Behaviour and implementation of the \textit{Lightning} exercise with different execution models.
        In the behaviour (top), marks represent when a new clone is made, while vertical lines indicate when the player either lost (dashed line) or completed 10 points (full line).
        The implementation (bottom) shows both variants of the user interaction code.
    }
    \label{fig:lightning-behaviour}
\end{figure}

The first project we looked at in detail is the \textit{Lightning} project (\cref{fig:the-game-lightning}).\footnote{\url{https://scratch.mit.edu/projects/995927372/}}
The aim of this game is to dodge lightning bolts coming out of the sky.
\Cref{fig:lightning-behaviour-original} gives an overview of the behaviour with the different execution models.
The \textsc{em-30},\textsc{em-60}, and \textsc{em-90} model did not achieve 10 points: they were too slow and made the game unplayable.

This is due to how the interactive component is implemented (\cref{fig:the-game-lightning-code-original}).
Adjusting the implementation by wrapping the user interaction blocks with a procedure (\cref{fig:the-game-lightning-code-better}) enabled solves this issue (\cref{fig:lightning-behaviour-procedure}).
This implementation is usable with all new execution models, except \textsc{em-30}, which remains too slow.

\minisec{\textit{Catch the apples} project}

\begin{figure}
    \centering
    \includestandalone{apple-scatter}
    \caption{
        Behaviour of the \textit{Catch the apples} exercise with different execution models.
        Marks represent score changes (or the initial score at 0), while vertical lines indicate when the player hits 10 points.
    }
    \label{fig:catch-apples-behaviour}
\end{figure}

The second project is the \textit{Catch the apples} project (\cref{fig:the-game-apple}).\footnote{\url{https://scratch.mit.edu/projects/995778768/}}
The aim here is to use the mouse to move a basket and catch as many falling apples as possible.
\Cref{fig:catch-apples-behaviour} shows the behaviour with different execution models.
In this exercise, all models are playable, with \textsc{em-30} behaving the most similar to the original Scratch execution model.
Models faster than \textsc{em-60} are likely too fast to actually play.

\subsection{Discussion}\label{subsec:discussion}

A few conclusions can be drawn from the results of this preliminary benchmark.

First, it is not obviously clear which variation of the execution model is the best universal replacement.
Different types of projects have different needs.
One solution for this problem would be to modify the virtual machine to change the frames per second depending on the type of project that is executed.

Secondly, projects depend on specific behaviour of the virtual machine.
For example, the \textit{Lightning} project's implementation of user interaction only works because of how the current execution model works.
While maybe unfortunate, as there are alternatives that do not depend on this behaviour, this is a consequence of \textit{Hyrum's Law}, which states that ``With a sufficient number of users, [\dots] all observable behaviours of your system will be depended on by somebody''.
Users, however, are already accustomed to changing the duration of wait blocks (and other such blocks) to account for the performance of the Scratch virtual machine on their device.
While not ideal, small differences may thus be acceptable.

Lastly, this illustrates the need for more research into automated evaluation of the behaviour exhibited by Scratch projects.
Some areas of interest are determining what constitutes observable behaviour and when changes to the behaviour become adverse changes.
For example, if projects are executed a bit faster or slower, this might not affect the project's usability.
The existing virtual machine is also not deterministic and depends on system performance: some projects might execute a lot slower on slower hardware, while still being usable.

\section{Impact and conclusion}\label{sec:conclusion}

In \cref{ch:blink}, we proposed a debugger for Scratch with a non-traditional step method.
Instead of stepping a single block at a time, we want to step a single block in every thread of a Scratch program (thus in every script).
However, this introduces two downsides:
\begin{enumerate*}[label=\emph{\roman*})]
    \item the step function exposes internal program state that is normally not visible to the users, and
    \item the debugger uses a different execution model compared to regular execution, whereas a debugger should deviate from normal execution as little as possible
\end{enumerate*}.

This chapter then asks if we can modify the execution model of Scratch in such a way that the step functionality of a debugger is possible and that the changed execution model is usable for normal execution.
To this end, we first took a detailed look at the existing execution model, due to a lack of existing literature on the topic.
This model is the result of multiple years of work in the Scratch virtual machine and contains many nuances.
The current execution model has been chosen to avoid some race conditions but does not avoid all concurrency-related issues: there are some surprising consequences of the threading model in particular.

Any changes to the execution model must not have adverse consequences for existing Scratch projects.
Therefore, we first analyse how Scratch is used by replicating select metrics of previous investigations into what Scratch projects look like.
The most significant previous result in this area is \textcite{aivaloglouHowKidsCode2016}, which analyses Scratch 2.0 projects.
Our results for Scratch 3.0 projects are broadly similar: most Scratch projects are small and simple, but there are a few big and complex ones.

Our proposed changes to the execution model change the threading from cooperative to preemptive.
This implies that more race conditions, which the original execution model sought to prevent, are now possible.
\Textcite{maloneyScratchProgrammingLanguage2010} use the classic example of reading the value of a variable, increasing this value, and finally updating the variable with the new value.
When running the same code unchanged in the new execution model, more race conditions are possible.
However, we believe this is not a big problem, due to two reasons.
First, the kind of code where these race conditions can occur is used infrequently in Scratch.

Second, there is a workaround: the part of the code that must be protected against these race conditions can be extracted into a procedure, using the option ``run without screen refresh''.
This causes the code in the function to become atomic: it will be run without interruptions.
Using this technique is our recommendation for implementing critical sections.
Note that using these critical sections with blocks that require a redraw should be done thoughtfully, as the redraw will not occur (this behaves identically in the current execution model).

Finally, we performed a preliminary benchmark using our proposed changes on a selection of projects, informed by our previous analysis.
The results of the benchmark make clear that finding a single replacement execution model without affecting existing projects is highly unlikely.
Projects depend on the behaviour of the current execution model, meaning any change in behaviour will be a breaking change.
Considering the Scratch Team's (understandable) reluctance to introduce behavioural changes to Scratch at this point, we do not envision our changes being upstreamed, nor would we recommend it, \marginnote{Scratch 4.0?} unless as a breaking change.

While a universally applicable replacement execution model is not achievable, our proposed changes are still useful.
For small projects, the impact of the changes is acceptable, and most projects are small.
Additionally, the new execution model is more suitable for use with our stepping method for debuggers.
We thus envision the new execution model to be used in classrooms where the debugger is used as well.

The results of the benchmarks are also a preliminary exploration.
Our proposed changes must still be validated in educational practice and in a classroom setting with actual users of Scratch, in additional to performing expanded automated benchmarks.
Another area of improvement is looking at more projects with user interaction for benchmarking.
One possible route is investigating heuristics to automatically provide suitable user interaction to those projects, making them suitable for automated benchmarking.
Improving the benchmarking allows for a better understanding on the impact of the proposed execution model.

We also see more opportunities for research on the existing and new execution model of Scratch itself.
For example, the current execution model is defined by its source code.
Constructing a formal mode of the execution model would allow formal reasoning and analysis, which might reveal more opportunities for changes.
This might also be beneficial in further analysing the new model's impact on existing Scratch projects.

\end{document}
