%! Suppress = MultipleIncludes
\documentclass[tikz,crop]{standalone}

\usepackage{amsmath}
\usepackage{tikz}
\usepackage{xcolor}
\usetikzlibrary{shapes,arrows,positioning,backgrounds,calc,intersections,calc}

\definecolor{ugent-re}{RGB}{220, 78, 40}        % vermilion			/ vermiljoen
\definecolor{ugent-we}{RGB}{45, 140, 168}        % no match
\definecolor{ugent-ge}{RGB}{232, 94, 113}        % rose				/ bleekrood
\definecolor{ugent-ea}{RGB}{111, 113, 185}        % distant blue		/ verblauw
\definecolor{ugent-pp}{RGB}{251, 126, 58}        % deep orange		/ dieporanje
\definecolor{ugent-ps}{RGB}{113, 168, 96}        % yellow green		/ geelgroen

\onlyifstandalone{
    \newlength{\unit}
    \setlength{\unit}{0.75cm}
}

\begin{document}

\tikzstyle{box}=[draw,rectangle,anchor=north west]
\tikzstyle{cont}=[box,minimum height=1\unit,minimum width=1\unit]
\tikzstyle{tc}=[box,draw=ugent-ps]
\tikzstyle{comp}=[box,draw=ugent-re,fill=ugent-re,fill opacity=0.05]
\tikzstyle{exec}=[box,draw=ugent-we,fill=ugent-we,fill opacity=0.10]

\begin{tikzpicture}[x=0.75cm,y=0.75cm]
    % \draw[step=1,gray!10,thin] (0,0) grid (14,-20);
    % Ensure width is the same
    \node[box,opacity=0,minimum width=13.5\unit] at (0,-1.75) (ts1) {};
%    \node[below right] at (ts1.north west) {Test suite};

    \node[cont] at (0.75, -1.75) (ts1c1) {C\textsubscript{1}};
    \node[cont] at (2.25, -1.75) (ts1c2) {C\textsubscript{2}};
    \node[cont] at (3.75, -1.75) (ts1c3) {C\textsubscript{3}};
    \node[tc,minimum height=2\unit,minimum width=4.5\unit] at (0.5,-1) (tc1) {};
    \node[below right,text=ugent-ps] at (tc1.north west) {\small Unit 1};

    \node[cont] at (5.50, -1.75) (ts1c4) {C\textsubscript{4}};
    \node[cont] at (7.00, -1.75) (ts1c5) {C\textsubscript{5}};
    \node[tc,minimum height=2\unit,minimum width=3\unit] at (5.25,-1) (tc2) {};
    \node[below right,text=ugent-ps] at (tc2.north west) {\small Unit 2};

    \node[cont] at (8.75, -1.75) (ts1c6) {C\textsubscript{6}};
    \node[cont] at (10.25, -1.75) (ts1c7) {C\textsubscript{7}};
    \node[cont] at (11.75, -1.75) (ts1c8) {C\textsubscript{8}};
    \node[tc,minimum height=2\unit,minimum width=4.5\unit] at (8.5,-1) (tc3) {};
    \node[below right,text=ugent-ps] at (tc3.north west) {\small Unit 3};

    \node[comp,minimum height=2.5\unit,minimum width=13\unit] at (0.25,-0.75) (comp) {};

    % Makes life much easier by shifting everything automatically.
    \begin{scope} [shift={(0,-2.75)}]
        \node[cont] at (0.75, -1.75) (ts1c1) {C\textsubscript{1}};
        \node[cont] at (2.25, -1.75) (ts1c2) {C\textsubscript{2}};
        \node[cont] at (3.75, -1.75) (ts1c3) {C\textsubscript{3}};
        \node[comp,minimum height=2\unit,minimum width=4.5\unit] at (0.5,-1) (tc1) {};
        \node[below right,text=ugent-ps] at (tc1.north west) {\small Unit 1};

        \node[cont] at (5.50, -1.75) (ts1c4) {C\textsubscript{4}};
        \node[cont] at (7.00, -1.75) (ts1c5) {C\textsubscript{5}};
        \node[comp,minimum height=2\unit,minimum width=3\unit] at (5.25,-1) (tc2) {};
        \node[below right,text=ugent-ps] at (tc2.north west) {\small Unit 2};

        \node[cont] at (8.75, -1.75) (ts1c6) {C\textsubscript{6}};
        \node[cont] at (10.25, -1.75) (ts1c7) {C\textsubscript{7}};
        \node[cont] at (11.75, -1.75) (ts1c8) {C\textsubscript{8}};
        \node[comp,minimum height=2\unit,minimum width=4.5\unit] at (8.5,-1) (tc3) {};
        \node[below right,text=ugent-ps] at (tc3.north west) {\small Unit 3};
    \end{scope}
\end{tikzpicture}

\end{document}
