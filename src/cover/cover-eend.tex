\documentclass[coverheight=240mm,coverwidth=170mm,spinewidth=15mm]{bookcover}

\usepackage{luacode}
\directlua{pdf.setminorversion(7)}

\usepackage{xcolor}
\usepackage[skip=0.5cm]{parskip}

% Original colours
\definecolor{istock-background}{HTML}{EDE6D6}
\definecolor{istock-foreground}{HTML}{72645E}

% New colours
\definecolor{ugent-we-dark}{HTML}{1F6276}
\definecolor{istock-bg-light}{HTML}{F2EDE2}

\definecolor{brownn}{HTML}{513214}

% What we actually use
\colorlet{cover-foreground}{brownn}
\colorlet{cover-background}{istock-bg-light}

\usepackage{pgfornament}

% Don't use the display type, as it is ugly.
% Similarly, don't use the caption type.
\usepackage{fontspec}
\setmainfont{Source Serif 4}[
    Color = cover-foreground,
    Renderer = OpenType,
    SizeFeatures = {
        {Size={-13},Font=*},
        {Size={14-},Font=* Subhead},
    },
    ItalicFeatures = {
        SizeFeatures = {
            {Size={-13},Font=* Italic},
            {Size={14-},Font=* Subhead Italic},
        },
    },
    BoldFeatures = {
        SizeFeatures = {
            {Size={-13},Font=* Semibold},
            {Size={14-},Font=* Subhead Semibold},
        },
    },
    BoldItalicFeatures = {
        SizeFeatures = {
            {Size={-13},Font=* Semibold Italic},
            {Size={14-},Font=* Subhead Semibold Italic},
        },
    },
    Numbers = OldStyle,
]

\usepackage{polyglossia}
\setdefaultlanguage[variant=british]{english}

\usepackage{microtype}
\usepackage{graphicx}
\usepackage{color}

\newenvironment{shadequote}[1]%
{
     \def\quoteauthor{#1}%
     \tikz[remember picture,overlay,xshift=-22pt]
     \node (OQ) {\fontsize{50pt}{50pt}\selectfont``};%
     \kern0pt\noindent%
}{%
    \par\hfill\textit{\fontsize{12pt}{15pt}\selectfont\quoteauthor}%
}

\begin{document}
\setbookcover{pagecolor=cover-background}
\begin{bookcover}
    \begin{bookcoverelement}{normal}{front}[10mm,15mm,10mm,15mm]%
        \centering
        \fontsize{34pt}{44pt}\selectfont DODONA \par
        \vspace*{-0.5cm} \par
        \tikz[baseline={([yshift=-5pt]current bounding box.south)}]{%
        \node[inner sep=0pt,color=cover-foreground]{\pgfornament[width=2cm]{80}};
        } \par
        \vspace*{0.1cm} \par
        \fontsize{30pt}{34pt}\selectfont IMPROVING PROGRAMMING EDUCATION \par
        \textit{\fontsize{20pt}{23pt}\selectfont through} \par
        \vspace*{0.3cm}
        \fontsize{20pt}{25pt}\selectfont AUTOMATED ASSESSMENT, LEARNING ANALYTICS, \textit{\fontsize{20pt}{23pt}\selectfont and} EDUCATIONAL DATA MINING \par
        \vspace*{1.5cm}
        \fontsize{20pt}{23pt}\selectfont\textit{\fontsize{15pt}{23pt}\selectfont by} \textsc{charlotte van petegem} \par
        \vspace*{\fill}
        \includegraphics[width=0.6\partwidth]{iStock-517349482} \par
        \vspace*{\fill}
        \fontsize{15pt}{18pt}\selectfont\textsc{ghent university} \par
        \fontsize{12pt}{15pt}\selectfont\textsc{mmxxiv}
    \end{bookcoverelement}
    \begin{bookcoverelement}{normal}{spine}[0mm,15mm,0mm,15mm]%
        \centering
        \rotatebox{-90}{%
            \fontsize{25pt}{17pt}\selectfont DODONA
        } \par
        \vspace*{\fill}
        \rotatebox[origin=c]{-90}{\fontsize{17pt}{22pt}\selectfont\textsc{charlotte van petegem}}
    \end{bookcoverelement}
%    \begin{bookcoverelement}{normal}{back}[25mm,15mm,25mm,25mm]
%        \fontsize{11pt}{15pt}\selectfont{}\setlength{\parskip}{3mm}
%
%        This dissertation proposes a set of new tools for improving automated assessment and debugging, aimed at enhancing both the learning and the teaching of programming.
%
%        For textual programming languages, we introduce \textsc{test}ed and \textsc{test}ed-\textsc{dsl}, two related frameworks designed to enable and streamline the authoring of programming exercises that can be used across programming languages.
%        The big benefit of this approach is that exercises become more reusable, and the more exercises can be used, the more exercises are available to students.
%
%        Young learns, however, have unique needs: they frequently used the block-based programming language Scratch, for which we introduce Itch and Blink.
%        Itch provides comprehensive testing capabilities for Scratch programs, while Blink offers an intuitive debugger, including time-traveling features.
%        These tools empower students to independently receive feedback and correct their mistakes, freeing teachers to focus on difficult cases, rather than spending time generating trivial feedback.
%        \vspace*{2cm}%
%
%        \fontsize{15pt}{17pt}\selectfont{}\setlength{\parskip}{3mm}
%        \begin{shadequote}{Steven V.}
%            Software, dat is toch niet publicatiewaardig?
%        \end{shadequote}
%
%        \begin{shadequote}{Kris K.}
%            Dit is geen doctoraat in de informatica.
%        \end{shadequote}
%
%        \vspace*{\fill}%
%        \begin{center}%
%            \pgfornament[scale=0.5,color=cover-foreground]{84}
%        \end{center}
%    \end{bookcoverelement}
%    \bookcovercomponent{ruler}{back}{,,}
%    \bookcovercomponent{ruler}{front}{,,}
    % \bookcovercomponent{ruler}{whole}{,,}
\end{bookcover}
\end{document}
