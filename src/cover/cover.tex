\documentclass[coverheight=240mm,coverwidth=170mm,spinewidth=15mm]{bookcover}

\usepackage{luacode}
\directlua{pdf.setminorversion(7)}

\usepackage{xcolor}
\usepackage[skip=0.5cm]{parskip}

% Original colours
\definecolor{istock-background}{HTML}{EDE6D6}
\definecolor{istock-foreground}{HTML}{72645E}

% New colours
\definecolor{ugent-we-dark}{HTML}{1F6276}
\definecolor{istock-bg-light}{HTML}{F2EDE2}

% What we actually use
\colorlet{cover-foreground}{ugent-we-dark}
\colorlet{cover-background}{istock-bg-light}

\usepackage{pgfornament}

% Don't use the display type, as it is ugly.
% Similarly, don't use the caption type.
\usepackage{fontspec}
\setmainfont{Source Serif 4}[
    Color = cover-foreground,
    Renderer = OpenType,
    SizeFeatures = {
        {Size={-13},Font=*},
        {Size={14-},Font=* Subhead},
    },
    ItalicFeatures = {
        SizeFeatures = {
            {Size={-13},Font=* Italic},
            {Size={14-},Font=* Subhead Italic},
        },
    },
    BoldFeatures = {
        SizeFeatures = {
            {Size={-13},Font=* Semibold},
            {Size={14-},Font=* Subhead Semibold},
        },
    },
    BoldItalicFeatures = {
        SizeFeatures = {
            {Size={-13},Font=* Semibold Italic},
            {Size={14-},Font=* Subhead Semibold Italic},
        },
    },
    Numbers = OldStyle,
]

\usepackage{polyglossia}
\setdefaultlanguage[variant=british]{english}

\usepackage{microtype}
\usepackage{graphicx}

%! suppress = DiscouragedUseOfDef
\newenvironment{shadequote}[1]%
{%
     \def\quoteauthor{#1}%
     \tikz[remember picture,overlay,xshift=-22pt]
     \node (OQ) {\fontsize{50pt}{50pt}\selectfont``};%
     \noindent%
}{%
    \par\hfill\textit{\fontsize{12pt}{15pt}\selectfont\quoteauthor}%
}

\begin{document}
\setbookcover{pagecolor=cover-background}
\begin{bookcover}
    \begin{bookcoverelement}{normal}{front}[15mm,15mm,15mm,15mm]%
        \centering
        \fontsize{34pt}{44pt}\selectfont EDUCATIONAL SOFTWARE TESTING \par
        \vspace*{0.5cm}
        \fontsize{18pt}{23pt}\selectfont\textit{for} \par
        \vspace*{0.5cm}
        \fontsize{27pt}{33pt}\selectfont TEXTUAL \textit{\fontsize{20pt}{23pt}\selectfont and} BLOCK-BASED PROGRAMMING LANGUAGES \par
        \vspace*{1.5cm}
        \fontsize{20pt}{23pt}\selectfont\textit{\fontsize{15pt}{23pt}\selectfont by} \textsc{niko strijbol} \par
        \vspace*{\fill}
        \includegraphics[width=0.65\partwidth]{iStock-ugent-we} \par
        \vspace*{\fill}
        \fontsize{15pt}{18pt}\selectfont\textsc{ghent university} \par
        \fontsize{12pt}{15pt}\selectfont\textsc{mmxxiv}
    \end{bookcoverelement}
    \begin{bookcoverelement}{normal}{spine}[0mm,15mm,0mm,15mm]%
        \centering
        \rotatebox{-90}{%
            \fontsize{10pt}{13pt}\selectfont\hspace*{0.4mm}for textual and block-based programming languages
        }%
        \hspace*{0.1cm}
        \rotatebox{-90}{%
            \fontsize{17pt}{22pt}\selectfont EDUCATIONAL SOFTWARE TESTING
        } \par
        \vspace*{\fill}
        \rotatebox[origin=c]{-90}{\fontsize{17pt}{22pt}\selectfont\textsc{niko strijbol}}
    \end{bookcoverelement}
    \begin{bookcoverelement}{normal}{back}[25mm,15mm,25mm,25mm]
        \fontsize{11pt}{15pt}\selectfont{}\setlength{\parskip}{3mm}

        This dissertation proposes a set of new tools for improving automated assessment and debugging, aimed at enhancing both the learning and the teaching of programming.

        For textual programming languages, we introduce \textsc{test}ed and \textsc{test}ed-\textsc{dsl}, two related frameworks designed to enable and streamline the authoring of programming exercises that can be used across programming languages.
        The big benefit of this approach is that exercises become more reusable, and the more exercises can be (re)used, the more exercises are available to students.

        Young learners, however, have unique needs not covered by textual languages.
        Hence, they frequently use the block-based programming language Scratch, for which we introduce Itch and Blink.
        Itch provides comprehensive testing capabilities for Scratch programs, while Blink offers an intuitive debugger, including time-traveling features.
        These tools empower students to independently receive feedback and correct their mistakes, freeing teachers from spending time generating trivial feedback.

        \vspace*{\fill}%
        \begin{center}%
            \pgfornament[scale=0.5,color=cover-foreground]{84}
        \end{center}
    \end{bookcoverelement}
%    \bookcovercomponent{ruler}{back}{,,}
%    \bookcovercomponent{ruler}{front}{,,}
    % \bookcovercomponent{ruler}{whole}{,,}
\end{bookcover}
\end{document}
