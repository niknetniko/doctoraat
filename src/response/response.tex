%! suppress = Makeatletter
%! Author = niko
%! Date = 5/10/2023

% Preamble
\documentclass[
    paper=a4,
    paper=portrait,
    parskip=half,
]{scrartcl}

\usepackage{luacode}
\directlua{pdf.setminorversion(7)}

\usepackage{fontspec}
\setmainfont{Source Serif 4}[
    Renderer = OpenType,
    SizeFeatures    = {%
        {Size={-9},Font=* Caption},
        {Size={9-13},Font=*},
        {Size={14-24},Font=* Subhead},
        {Size={24-},Font=* Display}
    },
    ItalicFeatures = {%
    SizeFeatures    = {%
            {Size={-9},Font=* Caption Italic},
            {Size={9-13},Font=* Italic},
            {Size={14-24},Font=* Subhead Italic},
            {Size={24-},Font=* Display Italic}
    },
    },
    BoldFeatures = {%
    SizeFeatures    = {%
            {Size={-9},Font=* Caption Semibold},
            {Size={9-13},Font=* Semibold},
            {Size={14-24},Font=* Subhead Semibold},
            {Size={24-},Font=* Display Semibold}
    },
    },
    BoldItalicFeatures = {%
    SizeFeatures    = {%
            {Size={-9},Font=* Caption Semibold Italic},
            {Size={9-13},Font=* Semibold Italic},
            {Size={14-24},Font=* Subhead Semibold Italic},
            {Size={24-},Font=* Display Semibold Italic}
    },
    },
    Numbers         = OldStyle,
]
\setsansfont{Source Sans 3}[
    UprightFont    = *-Regular,
    ItalicFont     = *-Italic,
    BoldFont       = *-Semibold,
    BoldItalicFont = *-Semibold Italic
]
\setmonofont{Source Code Pro}[
    Scale          = MatchLowercase,
    UprightFont    = *-Regular,
    ItalicFont     = *-Italic,
    BoldFont       = *-Semibold,
    BoldItalicFont = *-Semibold Italic
]
% Have a math font that at least works with our choices.
%\usepackage{unicode-math}
%\setmathfont{Erewhon Math}

% Use real upper and lower scripts.
\usepackage{realscripts}
\usepackage{microtype}


\usepackage{polyglossia}
\setdefaultlanguage[variant=british]{english}

\author{Niko Strijbol}
\title{Response to the reading reports for doctoral dissertation.}

\begin{document}

\maketitle

We would like to thank the reading committee of the jury for taking the time and effort to read the dissertation and for their insightful and helpful comments on the text.
In this document, we detail the changes we have made to the text, or why we have not made some changes.
The document here is a summary of the changes: the detailed changes are better viewed in the provided diff with the previous version of the text.

\begin{enumerate}
    \item Various typos and unclear sentences have been fixed and adjusted.
        We again want to thank the jury members for pointing these out.
    \item The publication list has been split to better indicate which of the publications are directly related to this dissertation and which are not.
    \item The subsection on execution planning (page 36) has been changed to hopefully make it clearer.
    \item Chapter 8, with the conclusions, has been slightly re-structured to make it less awkward.
    \item Some data (e.g.\ about Scratch) were updated to use the most recently available data (July and August instead of April and May).
        This has no impact on any conclusions or other text.
    \item We have summarized the three quasi-experiments of chapter 2 (TESTed) in a table, which makes the data easier to read.
    \item Figure 6.2 has been reworked as a sequence diagram to better show the interaction between the components.
    \item Whether to capitalize references to figures, tables, etc.\ in the text (Figure X versus figure X) is a personal choice, and opinions differ on the subject.
    For example, the \textit{APA Style Guide} recommends capitalization, but others, like the \textit{Chicago Manual of Style} and \textit{Hart's rules}, recommend not capitalizing the references.
    We have thus not applied the suggestion to capitalize the references.
    We have applied the other suggestion about putting footnote marks after punctuation.
    \item It is suggested that chapter 2 (TESTed) could use more emphasis on its different uses (in regard to the support for black box and unit testing).
    Considering the numerous existing frameworks that support black-box testing, chapter 3 (which has more examples), and the length of the dissertation, we decided to keep the focus of the examples on unit testing in this chapter.
    We have improved some references to chapter 3. (TODO)
    \item For the suggestion to include a bit more related work on programming course management systems, we have chosen to explicitly point the reader to some other review papers for an overview.
\end{enumerate}


\end{document}
