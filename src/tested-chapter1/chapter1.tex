%! Author = niko
%! Date = 11/01/2024

\documentclass[../main]{subfiles}
\usepackage{glossaries}

% Document
\begin{document}

\chapter{Testing frameworks in educational contexts}
\label{ch:testing-frameworks-in-educational-contexts}

Te doen: wat introductie over dit deel van de thesis.

In this chapter, we discuss software testing in an educational context.
This serves as an introduction to the next chapters, which discuss TESTed, our own educational testing framework.
We begin by giving a short overview of software testing, followed by how testing is used in an educational context.
Next, we identify the challenges unique to the eductional context.
We also look at existing and related work for these challenges.

\section{Software testing}
\label{sec:software-testing}

Software testing is the validation and verification of a system derived from source code (Ammann \& Offutt, 2016).
Two complementary approaches prevail: dynamic testing executes the source code with a given suite of test cases, whereas static testing analyzes the source code without executing it (Romli et al., 2010). Both approaches might be done via manual or automated processes, based on a specification (Pieterse, 2013). In modern software development, automated testing has become a standard practice for continuous integration and continuous development of living codebases, where both the code base and the system requirements might evolve over time (Winters et al., 2020). The minimum requirement that is tested is correctness, which is the essential purpose of software testing (Pan, 1999b). Automated testing for correctness needs some kind of oracle to tell the right behavior from the wrong one. Other software quality factors that may be tested are its functionality (reliability, usability, integrity), engineering (efficiency, testability, documentation, structure) and adaptability (flexibility, reusability, maintainability) (Hetzel, 1988).


\end{document}
