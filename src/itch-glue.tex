\documentclass[./main]{subfiles}

\begin{document}

\chapter*{Preface}

This part of the thesis deals with tools for and research into Scratch, a visual programming language targeted at young users.
The part begins (\cref{ch:scratch-the-programming-language}) with a short chapter giving an introduction of Scratch, rather than each chapter including the same introduction.

The work on Scratch was started as a collaboration between Ghent University and CodeCosmos.
CodeCosmos, the international brand of FTRPRF, is an educational publisher that creates teaching packs that schools can use to fulfill their obligations as part of the move to introduce more computational thinking in secondary education.

The work was divided as follows: Ghent University was responsible for the technical aspects of the automated feedback for Scratch exercises, while CodeCosmos provided the lessons, the educational support and, last but not least, actual students to use the teaching packs.
Readers interested in knowing more might like to read an article about this collaboration in \textit{Dare To Think}, Ghent University's online magazine~\autocite{daretothinkCoachCoDiMotivationboosting2023}.

The testing framework created as a result of this collaboration is called Itch, described in \cref{ch:itch}.
It is an important part of the ``Coach CoDi'' tool in the platform of CodeCosmos.

The experience of using Itch in the classroom and with students made clear that a testing framework helps to know when an exercise has been more or less satisfactorily solved.
However, it does very little to help students find the problem in their code when they are stuck.

At that point, work began on investigating a debugger for Scratch: Blink.
\Cref{ch:blink}, which describes this debugger, is based on a publication:

\begin{itemize}
    \item \fullcite{strijbolBlinkEducationalSoftware2024}
\end{itemize}

Work on Blink was the direct cause for \cref{ch:scratch-execution-model}.
The current execution model of Scratch has made some choices that limit how a debugger can work.
In this last chapter, we investigate a series of modifications to the execution model and look at the consequences of those changes.

\end{document}
