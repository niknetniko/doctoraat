\documentclass[./main]{subfiles}

\begin{document}

\addchap{Preface to the second part}\label{ch:preface-2}

This part of the dissertation deals with tools for and research into Scratch, a visual block-based programming language targeted at young users.
We begin with \cref{ch:scratch-the-programming-environment}, which gives an introduction to Scratch.

The first prototype of the testing framework, called Itch and discussed in \cref{ch:itch}, was first created at Ghent University.
Then, in an effort to transform this prototype to usable software, we started a collaboration with CodeCosmos.
CodeCosmos, the international brand of FTRPRF, is an educational publisher that creates teaching packs that schools can use to fulfil their obligations as part of the move to introduce more computational thinking in secondary education.
The testing framework is an important part of the ``Coach CoDi'' tool in the platform of CodeCosmos.

The work was divided as follows: Ghent University was responsible for the technical aspects of the automated feedback for Scratch exercises, while CodeCosmos provided the lessons, the educational support, and last but not least, actual students to use the teaching packs.
Readers interested in knowing more might like to read an article about this collaboration in \textit{Dare To Think}, Ghent University's online magazine\footnote{\url{https://www.durfdenken.be/en/research-and-society/coach-codi-boosting-tool-helps-children-become-independent-coders}}.

The experience of using Itch in the classroom and with students made clear that a testing framework helps to know when an exercise has been satisfactorily solved.
However, it does little to help students find the problem in their code when they are stuck.
At that point, work began on investigating a debugger for Scratch: Blink.
\Cref{ch:blink}, which describes this debugger, is based on:

\begin{itemize}
    \item \fullcite{strijbolBlinkEducationalSoftware2024}
\end{itemize}

Work on Blink was the cause for looking into the Scratch execution model (\cref{ch:scratch-execution-model}).
Its current threading model is designed to prevent some problems associated with concurrency, like race conditions.
However, not all problems are prevented.
Additionally, the execution model also limits how a debugger can work, without resorting to deviating from the execution model.
In this last chapter, we investigate a series of modifications to the execution model and look at the consequences of those changes.

\end{document}
