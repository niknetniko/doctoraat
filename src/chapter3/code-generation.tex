%! Suppress = MultipleIncludes
\documentclass[tikz]{standalone}

\onlyifstandalone{
    \usepackage[outputdir=../../out/]{minted}
    \setminted{autogobble=true}
}

% Part of the preamble, for TikZ figures.
% This is used in both the main document and in the subfigures.
% One exception is minted: since the path depends on the file, it is not set.
\usepackage{tikz}
\usepackage{xcolor}
\usepackage{pgfplots}

\pgfplotsset{compat=1.18}
\usepgfplotslibrary{statistics}

\usetikzlibrary{shapes,arrows,positioning,backgrounds,calc,intersections,calc}

\definecolor{ugent-re}{RGB}{220, 78, 40}        % vermilion			/ vermiljoen
\definecolor{ugent-we}{RGB}{45, 140, 168}       % no match
\definecolor{ugent-ge}{RGB}{232, 94, 113}       % rose				/ bleekrood
\definecolor{ugent-ea}{RGB}{111, 113, 185}      % distant blue		/ verblauw
\definecolor{ugent-pp}{RGB}{251, 126, 58}       % deep orange		/ dieporanje
\definecolor{ugent-ps}{RGB}{113, 168, 96}       % yellow green		/ geelgroen

\tikzstyle{python}=[fill=ugent-ps!50!white]
\tikzstyle{java}=[fill=ugent-we!50!white]
\tikzstyle{haskell}=[fill=ugent-ea!50!white]
\tikzstyle{js}=[fill=ugent-pp!50!white]
\tikzstyle{c}=[fill=ugent-re!50!white]

\newlength{\block}
\setlength{\block}{0.75cm}

\tikzstyle{a}=[anchor=north west]
\tikzstyle{box}=[a,draw,rectangle]
\tikzstyle{node}=[a,draw,minimum height=0.5cm,align=center,fill=white,text depth=.25ex]
\tikzstyle{document}=[node,tape,tape bend top=none]
\tikzstyle{cont}=[box,minimum height=1\block,minimum width=1\block]
\tikzstyle{arrow}=[draw, -latex]
\tikzstyle{inner}=[box,draw=gray]

% Blue box style
\tikzstyle{bluebox}=[draw=ugent-we,java]
\tikzstyle{redbox}=[draw=ugent-re,c]
\tikzstyle{greenbox}=[draw=ugent-ps,python]

% Some things specific to TESTed imagery.
\tikzstyle{tc}=[box,draw=ugent-ps]
\tikzstyle{comp}=[box,draw=ugent-re,fill=ugent-re,fill opacity=0.05]
\tikzstyle{exec}=[box,draw=ugent-we,fill=ugent-we,fill opacity=0.10]

% Stuff from tested-engine/concept.tex
\tikzstyle{process}=[node,rectangle]
\tikzstyle{terminator}=[node,rectangle,rounded corners=0.5cm]
\tikzstyle{io}=[node,trapezium,trapezium left angle=70,trapezium right angle=-70,minimum width=2.5cm,trapezium stretches=true]
\tikzstyle{small}=[font=\footnotesize,color=darkgray]
\tikzstyle{submission}=[document,align=right,minimum width=3cm,minimum height=1cm,text depth=0.5cm,inner sep=0.5mm,font=\scriptsize]

% Stuff from chatper3/flow.tex
\tikzstyle{height}=[minimum height=0.75\block]
\tikzstyle{contt}=[cont,minimum height=0.75\block]
\tikzstyle{compop}=[comp,text opacity=1]
\tikzstyle{execop}=[exec,text opacity=1]

\tikzstyle{hnode}=[draw,anchor=center,minimum height=\block,text depth=.25ex,align=center]
\tikzstyle{executable}=[hnode,ultra thick,fill=gray!10]
\tikzstyle{inner-exec}=[node,anchor=center,minimum width=3.25\block,densely dotted,font=\footnotesize,fill=none]
\tikzstyle{stmt}=[node,anchor=center,fill=gray!30,minimum width=4.5\block,font=\footnotesize]
\tikzstyle{fieldset}=[minimum height=\block,fill=white,text depth=.5ex,fill=white]

% Minted environments for use in Tikz
\newminted[tikzjava]{java}{autogobble,linenos=false,fontsize=\tiny,stripall}
\newminted[tikzpython]{python}{autogobble,linenos=false,fontsize=\tiny,stripall}
\newminted[tikztext]{text}{autogobble,linenos=false,fontsize=\tiny,stripall}


\begin{document}

\begin{tikzpicture}[x=0.75cm,y=0.75cm]
    \begin{scope} [shift={(1.1125,1.5)}]
        \node[cont] at (0.75, -1.75) (ts1c1) {C\textsubscript{1}};
        \node[cont] at (2.25, -1.75) (ts1c2) {C\textsubscript{2}};
        \node[cont] at (3.75, -1.75) (ts1c3) {C\textsubscript{3}};
        \node[tc,minimum height=2\unit,minimum width=4.5\unit] at (0.5,-1) (tc1) {};
        \node[below right,text=ugent-ps] at (tc1.north west) {\small Unit 1};

        \node[cont] at (5.50, -1.75) (ts1c4) {C\textsubscript{4}};
        \node[cont] at (7.00, -1.75) (ts1c5) {C\textsubscript{5}};
        \node[tc,minimum height=2\unit,minimum width=3\unit] at (5.25,-1) (tc2) {};
        \node[below right,text=ugent-ps] at (tc2.north west) {\small Unit 2};

        \node[cont] at (8.75, -1.75) (ts1c6) {C\textsubscript{6}};
        \node[cont] at (10.25, -1.75) (ts1c7) {C\textsubscript{7}};
        \node[cont] at (11.75, -1.75) (ts1c8) {C\textsubscript{8}};
        \node[tc,minimum height=2\unit,minimum width=4.5\unit] at (8.5,-1) (tc3) {};
        \node[below right,text=ugent-ps] at (tc3.north west) {\small Unit 3};

        \node[comp,minimum height=2.5\unit,minimum width=13\unit] at (0.25,-0.75) (comp) {};

        \node[exec,minimum height=2.25\unit,minimum width=4.25\unit] at (0.625,-0.875) (execu1) {};
        \node[exec,minimum height=2.25\unit,minimum width=1.25\unit] at (5.375,-0.875) (execu2) {};
        \node[exec,minimum height=2.25\unit,minimum width=6\unit] at (6.875,-0.875) (execu3) {};
    \end{scope}

    \begin{scope} [shift={(0,-3)}]
        \node[document,minimum width=15.75\unit, minimum height=8.25\unit,draw=ugent-re] at (0,0) (executable) {};
        \node[below right,text=ugent-re] at (executable.north west) {\small Executable};

        \node[box,draw=ugent-re,minimum height=3.25\unit,minimum width=3.5\unit] at (0.25,-0.75) (main) {};
        \node[below right,text=ugent-re] at (main.north west) (mtitle) {\small \texttt{main} function};
        \node[below right,text width=3.75\unit] at (mtitle.south west) {%
            \fvset{listparameters=\setlength{\topsep}{0pt}\setlength{\partopsep}{0pt}}%
            \begin{tikzjava}
                switch(argument) {
                  case "1" -> exec1();
                  case "2" -> exec2();
                  case "3" -> exec3();
                  default -> throw ...;
                }
            \end{tikzjava}
        };

        \node[exec,minimum height=3.25\unit,minimum width=3.5\unit] at (4,-0.75) (exec1) {};
        \node[below right,text=ugent-we] at (exec1.north west) (e1title) {\small \texttt{exec1} function};
        \node[below right,text width=3.75\unit] at (e1title.south west) {%
            \fvset{listparameters=\setlength{\topsep}{0pt}\setlength{\partopsep}{0pt}}%
            \begin{tikzjava}
                outputContextSecret();
                context1();
                outputContextSecret();
                context2();
                outputContextSecret();
                context3();
            \end{tikzjava}
        };

        \node[exec,minimum height=3.25\unit,minimum width=3.5\unit] at (7.75,-0.75) (exec2) {};
        \node[below right,text=ugent-we] at (exec2.north west) (e2title) {\small \texttt{exec2} function};
        \node[below right,text width=3.75\unit] at (e2title.south west) {%
            \fvset{listparameters=\setlength{\topsep}{0pt}\setlength{\partopsep}{0pt}}%
            \begin{tikzjava}
                outputContextSecret();
                context4();
            \end{tikzjava}
        };

        \node[exec,minimum height=3.25\unit,minimum width=3.5\unit] at (11.5,-0.75) (exec3) {};
        \node[below right,text=ugent-we] at (exec3.north west) (e3title) {\small \texttt{exec3} function};
        \node[below right,text width=3.75\unit] at (e3title.south west) {%
            \fvset{listparameters=\setlength{\topsep}{0pt}\setlength{\partopsep}{0pt}}%
            \begin{tikzjava}
                outputContextSecret();
                context5();
                outputContextSecret();
                context6();
                outputContextSecret();
                context7();
                outputContextSecret();
                context8();
            \end{tikzjava}
        };

        % Normally, context functions should be here, however, they are drawn at the end.
    \end{scope}


    \draw[-latex,draw=ugent-we] (execu1) to [out=-90,in=90] (exec1);
    \draw[-latex,draw=ugent-we] (execu2) to [out=-90,in=90] (exec2);
    \draw[-latex,draw=ugent-we] (execu3) to [out=-90,in=90] (exec3);

    \draw[-latex,draw=ugent-re] (comp) -- (executable) node[midway,fill=white,fill opacity=0.8,text opacity=1] {\small Code generation and compilation};

    \begin{scope} [shift={(0,-12)}]
        \node[document,draw=ugent-we,minimum width=4.75\unit, minimum height=3.25\unit] at (0.25, 0) (res1) {};
        \node[below right,color=ugent-we] at (res1.north west) (res1title) {\small \texttt{output1.txt}};
        \node[below right,text width=3.75\unit] at (res1title.south west) {%
            \fvset{listparameters=\setlength{\topsep}{0pt}\setlength{\partopsep}{0pt}}%
            \begin{tikztext}
                context_separator (C1)
                testcase_separator
                result1
                testcase_separator
                result2
            \end{tikztext}
        };

        \node[document,draw=ugent-we,minimum width=4.75\unit, minimum height=3.25\unit] at (5.5, 0) (res2) {};
        \node[below right,color=ugent-we] at (res2.north west) (res2title) {\small \texttt{output2.txt}};
        \node[below right,text width=3.75\unit] at (res2title.south west) {%
            \fvset{listparameters=\setlength{\topsep}{0pt}\setlength{\partopsep}{0pt}}%
            \begin{tikztext}
                context_separator (C1)
                testcase_separator
                result1
                context_separator (C2)
                testcase_separator
                result1
                ...
            \end{tikztext}
        };

        \node[document,draw=ugent-we,minimum width=4.75\unit, minimum height=3.25\unit] at (10.75, 0) (res3) {};
        \node[below right,color=ugent-we] at (res3.north west) (res3title) {\small \texttt{output3.txt}};
        \node[below right,text width=3.75\unit] at (res3title.south west) {%
            \fvset{listparameters=\setlength{\topsep}{0pt}\setlength{\partopsep}{0pt}}%
            \begin{tikztext}
                ...
                context_separator (C8)
                testcase_separator
                result1
            \end{tikztext}
        };
    \end{scope}

    \draw[-latex,draw=ugent-we] (exec1) to [out=-90,in=90] (res1);
    \draw[-latex,draw=ugent-we] (exec2) to [out=-90,in=90] (res2);
    \draw[-latex,draw=ugent-we] (exec3) to [out=-90,in=90] (res3);

    % Context functions, to be before the arrows
    \begin{scope} [shift={(0,-3)}]
        \node[cont,minimum width=4\unit, minimum height=3\unit,fill=white] at (0.25, -4.75) (cf1) {};
        \node[below right] at (cf1.north west) (c1title) {\small \texttt{context1} function};
        \node[below right,text width=3.75\unit] at (c1title.south west) {%
            \fvset{listparameters=\setlength{\topsep}{0pt}\setlength{\partopsep}{0pt}}%
            \begin{tikzjava}
                outputTestcaseSeparator();
                testcase1();
                outputTestcaseSeparator();
                testcase2();
                outputTestcaseSeparator();
                testcase3();
                ...
            \end{tikzjava}
        };

        \node[cont,minimum width=4\unit, minimum height=3\unit,fill=white] at (5, -4.75) (cf2) {};
        \node[below right] at (cf2.north west) (c2title) {\small \texttt{context2} function};
        \node[below right,text width=3.75\unit] at (c2title.south west) {%
            \fvset{listparameters=\setlength{\topsep}{0pt}\setlength{\partopsep}{0pt}}%
            \begin{tikzjava}
                outputTestcaseSeparator();
                testcase1();
                outputTestcaseSeparator();
                testcase2();
            \end{tikzjava}
        };

        \node[cont,draw opacity=0,align=center] at (9.75, -4.75) {\ldots};

        \node[cont,minimum width=4\unit, minimum height=3\unit,fill=white] at (11.5, -4.75) (cf8) {};
        \node[below right] at (cf8.north west) (c8title) {\small \texttt{context8} function};
        \node[below right,text width=3.75\unit] at (c8title.south west) {%
            \fvset{listparameters=\setlength{\topsep}{0pt}\setlength{\partopsep}{0pt}}%
            \begin{tikzjava}
                outputTestcaseSeparator();
                testcase1();
            \end{tikzjava}
        };
    \end{scope}

    \draw[draw opacity=0] (executable) -- (res2) node[midway,fill=white,fill opacity=0.8,text opacity=1] {\small Execution};

\end{tikzpicture}

\end{document}
