\documentclass[main]{subfiles}

\begin{document}

\selectlanguage{dutch}

\chapter{Dankwoord}\label{ch:dankwoord}

Het gebeurt wel eens dat een dankwoord van een doctoraatsproefschrift begint met de vermelding dat het om een werk van lange adem gaat.
Dat is echter niet hoe ik het ervaren heb: aan de start van mijn doctoraat leken de vier jaar een eeuwigheid, maar nu heb ik het gevoel dat alles heel snel gegaan is.
\marginnote{Ik negeer dat de tijd ook sneller gaat naarmate we ouder worden.}
Ongetwijfeld heeft de steun van collegae, vrienden en familie hier een groot aandeel in: iedereen weet dat de tijd sneller gaat als het leuk is.
Dat brengt ons bij het doel dezes paragraaf: mijn dank uiten aan een hele hoop mensen.

Allereerst wil ik mijn promotoren bedanken: prof.\ Peter Dawyndt, prof.\ Bart Mesuere en prof.\ Christophe Scholliers.
Van alle mensen vermeld in dit dankwoord zijn zij waarschijnlijk degenen die de inhoudelijkste bijdragen geleverd hebben aan dit proefschrift.
Dit gaat dan van de in regel wekelijkse vergaderingen waar we soms kort, soms lang bepaalde onderwerpen bespraken, tot de vele suggesties en opmerkingen over verschillende artikels (en dit proefschrift).
Bedankt voor het mogelijk maken van dit doctoraat, en de voortreffelijke begeleiding en steun die ik hierbij gekregen heb.

Ik wil ook graag mijn juryleden (de ``examencommissie'') bedanken: prof.\ Bart Dhoedt, prof.\ Veerle Fack, prof.\ Gordon Fraser\marginnote{Special thanks to Gordon for coming all the way to Ghent for my internal defence.}, and prof.\ Frank Neven.
Een speciale bedanking ook aan prof.\ Chris Cornelis om de jury te willen voorzitten.

Delen van dit proefschrift zijn tot stand gekomen in samenwerking met FTRPRF/CodeCosmos.
Hoewel er soms iets misliep, zoals de judge die plots stopt met werken, ben ik trots op wat we samen verwezenlijkt hebben.
Ik wil dan ook de mensen met wie ik heb samengewerkt bedanken: Katelijne Duerinck, Kristien Duerinck, Peter Keyngnaert, Cedric Vanhaverbeke, Elisa David, Morgane Kruglanski, Jannick Vandaele en Glenn Thielman.

Hoewel de meeste collega's geen rechtstreekse\marginnote{Als de soms wat te lange koffiepauzes in overweging genomen worden is de bijdrage misschien soms negatief.} bijdragen geleverd hebben aan dit proefschrift, zorgden ze door een toffe werksfeer (en ook buiten het werk, zoals met het \textsc{twi}kend) ervoor dat ik zin had om te blijven werken aan de Universiteit Gent.
De lange lijst is Alexis Langlois-Rémillard, Asmus Bilbo, Benjamin Rombaut, Charlotte Van Petegem, Felix Van der jeugt, Heidi Van den Camp, Jonathan Peck, Jorg Van Renterghem, Louise Deconinck, Mustapha Regragui, Oliver Urs Lenz, Pieter Verschaffelt, Rien Maertens, Robbert Gurdeep Singh, Tibo Vande Moortele, Tom Lauwaerts, Toon Baeyens, Simon Reyntjens en Steven Van Overberghe.
Veel van deze mensen waren al of zijn ondertussen ook vrienden geworden.
Ik wil ook Annick Van Daele bedanken, voor het leuk maken van de werkcolleges Programmeren, samen met Charlotte en Toon.

Hoe verder in dit dankwoord, hoe verder van het bijdragen aan het proefschrift \textit{in se} dat ik ga, waardoor ik nu aangekomen ben bij de vrienden die ik tijdens mijn opleiding heb leren kennen: Arne Gevaert, Jarre Knockaert, Jorg Van Renterghem, Louise Deconinck, Nils Mak, Pieter Verschaffelt, Sam Persoon, Sander Vanhove, ondertussen aangevuld met Chloë, Charlotte (Parmentier), Ciel en Liesbeth.
Bedankt voor de vele leuke activiteiten we samen doen.

Pieter, ik denk dat ik niet anders kan dan je nog eens apart te vermelden.
Sinds praktisch het begin van de opleiding, nu al tien jaar geleden, hebben we heel veel samen gedaan.
Dit waren soms kleine dingen, zoals groepswerken, keuzevakken, examentoezichten of gaan eten bij de Griek.
Soms waren het grotere dingen, zoals als we eens op reis gaan.
Ik denk dat onze reis naar Malta en onze trektocht door Bulgarije, Griekenland en Noord-Macedonië (samen met Jarre) me nog lang gaan bij blijven.
Ik ben je heel dankbaar voor al deze momenten, en ondanks dat we nu voor het eerst in tien jaar niet meer samen studeren of werken, weet ik dat er nog veel leuke momenten gaan volgen.

Nog wil ik de vele leden van Zeus\textsc{wpi} bedanken, waar ik tijdens de opleiding veel plezier aan gehad heb\marginnote{Hydra is mijn langstlopende project aan de universiteit.}, en tijdens de eerste jaren van mijn doctoraat bij het thuiswerken vanwege de coronapandemie veel uren heb doorgebracht op Mattermost.
De lijst van leden die ik zou kunnen bedanken is zo lang, dat ik er zelfs niet aan durf te beginnen.
Daarnaast wil ik graag Wout Schellaert bedanken om mij te overtuigen een studentenbaan aan te nemen bij de Dienst StudentenActiviteiten.
Ook alle medewerkers die daar destijds werkten wil ik bedanken, met een speciale vermelding van Nicolas Vander Eecken.

Tot slot, \textit{last but not least}, wil ik mijn familie bedanken: mama, papa, Eliah en Maya, die mij allemaal wat meer of minder geholpen hebben tijdens mijn doctoraat, hetzij door te helpen verhuizen, hetzij door een luisterend oor te zijn als het even minder ging, hetzij door met mij op reis te gaan.
Mijn ouders in het bijzonder ook om mij alle kansen te geven en mij te steunen in de keuzes die ik maak, zoals de keuze om informatica te studeren.
Bedankt!

\pagebreak

Een proefschrift als het mijne is vaak zeer eenzijdig: ik, als auteur, vertel allerlei dingen en van de lezer verwacht ik hoogstens wat denkwerk.
Hoog tijd om daar verandering in te brengen!
In onderstaande woordzoeker komt elke voornaam van elke vermelde persoon uit dit dankwoord voor.
Als oplossing geeft hij nog mijn laatste boodschap aan de lezer voor de echte inhoud begint.

\begin{filecontents*}[overwrite]{letters.txt}
V E E B A R T N S S L M A M A C V
P P T U O W O T I I A S M U S E L
E G E Z I O E M E R X M I M E D E
E O R T T V O T R A B E N R A R F
L B R O E N T X I L E F L H H I R
I I A N J R L E T Z E E N A E C A
S T J S A N D E R R N N N I S M N
A N I M A J N E B E E I N L I U K
E V N Y A A N M H L C B L E U S E
N IJ A E G R N P G O I H B S O T T
IJ M P R I E O R L J G E L O L A T
L J O N A T H A N O A O S O R P O
E M E F S E S S R C H N R B E H L
T O R I I I I I P A P A N D E A R
A T R D F P E T R E T E P I O T A
K H I A N N I C K K L E I C C N H
C H A R L O T T E R E V I L O K C
\end{filecontents*}

\readdef{letters.txt}\lettersdata
\readarray*\lettersdata\letters[-,17]

\vspace{\fill}

\begin{wide}
    \begin{tikzpicture}[x=0.75cm,y=0.75cm]
        % Draw internal grid.
%        \draw[step=0.75cm,color=gray] (1,1) grid (18,18);
        % Draw nodes
        \foreach \row in {1,...,17} {
            \foreach \col in {1,...,17}{
                \node[draw=gray,align=center,minimum width=0.75cm,minimum height=0.75cm] (\col\row) at (\col,-\row) {\letters[\row,\col]};
            }
        }

        \draw[line width=0.5cm,ugent-we,opacity=0.5,line cap=round] (12,-1) -- (15,-1);
        \draw[line width=0.5cm,ugent-we,opacity=0.5,line cap=round] (9,-14) -- (12,-14);
    \end{tikzpicture}
\end{wide}

{\small Een kleine aanwijzing: er worden 51 mensen vermeld in dit dankwoord en het antwoord bestaat uit 37 letters.}



\selectlanguage{english}
\end{document}
