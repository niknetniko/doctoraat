%! Author = niko
%! Date = 18/10/2023

% Preamble
\documentclass[main]{subfiles}
\usepackage{glossaries}

% Document
\begin{document}

\chapter{Itch: an automated test framework for Scratch projects}
\label{ch:itch:-an-automated-test-framework-for-scratch-projects}


% TODO: Korte introductie, afhankelijk van wat er in de rest gaat staan

\section{Executing tests with Itch}\label{sec:executing-tests-with-itch}

\newacronym{put}{put}{project under test}

To test a project with Itch, three input files are required:

\begin{enumerate}
    \item The project which will be tested, the \acrfull{put}.
    \item A project containing a sample solution.
          If this is not available, it may be substituted by the \acrshort{put}.
    \item A test suite.
\end{enumerate}

Itch will then execute the test suite on the \acrshort{put} and output the results.
A test has three phases: before execution, during execution and after execution.


\section{Test suite}\label{sec:test-suite}

Tests for Itch are written in JavaScript.

\section{Test phases}\label{sec:test-phases}

\subsection{Before execution}\label{subsec:before-execution}

The phase before execution is run before the Scratch VM is run (and is thus aptly named).
The main purpose of the phase is to execute static tests.
It is also in the phase that the sample project is used, as it might be used in comparison to the \acrshort{put}.

While all manners of static testing are possible, Itch provides specific support for two scenarios.

First, the scenario where there is predefined code in the Scratch project.
For example, students may receive a template to start from, in which certain sprites or parts of sprites are already implemented.
In most cases, changing this code will break the project.
Itch thus contains functionality to test if the predefined code in the submission still matches with the code in the sample solution.
To accommodate different scenarios, one can specify which sprites must be the same, but also which scripts of a sprite must be the same.
Besides blocks, similar functions exist to check that sprites have not changed.

Secondly, Itch provides a Scratch block pattern matching library.
This allows for checks that the code in the submission exactly matches a pattern in the test suite.
This is mainly useful in exercises where the aim is to work on a specific construct, e.g.\ loops or control flow.

\subsection{During execution (the scheduler)}\label{subsec:during-execution-(the-scheduler)}

Scratch projects are often executed interactively, meaning they require interaction to run.
For example, pressing the green flag starts the project, or clicking on a sprite makes the sprite dio something.






\section{Execution environments}\label{sec:execution-environments}



\end{document}
