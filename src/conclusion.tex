\documentclass[main]{subfiles}

\begin{document}

\chapter{Conclusions and opportunities}\label{ch:conclusions-and-opportunities}

In this work, we introduce five educational tools for learning programming, two for textual and three for block-based programming languages.
While using Dodona, we noticed some gaps in existing educational tools for programming education.
We developed the presented tools as a response to this, attempting to fill the gap.

%\marginnote{This chapter might include ``forward-looking'' statements.}

\section{For textual programming languages}\label{sec:for-textual-programming-languages}

For block-based languages, we first identify input/output testing and unit testing as two opposing strategies commonly used in educational software testing.
We first investigated the impact of both approaches on programming language support of the frameworks.
Then, we also looked at the effect of the strategies on the software testing itself, like the impact on the feedback or what can be tested.

Our goal was to combine the best of both worlds.
We formulated requirements for programming-language-agnostic testing frameworks that combine unit testing with support for multiple programming languages.
We see three clear benefits for the adoption of such frameworks: \begin{enumerate*}[label=\emph{\roman*})] \item sharing the same declarative structure across programming languages, \item bridging the gap between input/output testing and unit testing, and \item allowing test code to be expressed in a language-agnostic way.\end{enumerate*}

Our goal is to further develop TESTed as an educational software testing framework for authoring different types of programming exercises across programming languages.
TESTed is currently focussed on dynamic testing.
However, of future interest is the introduction of a language-agnostic way to perform static code analysis.

We are also investigating extensions to TESTed-DSL, such as operators for testing operator overloading, string conversion, comments, indexing sequences, indexing mappings, destructuring, object identity checking, and object equivalence checking.
Additionally, an interesting area for future research is native support for pretty printing of nested data structures to make it easier to detect differences between expected and actual return values.
There are more opportunities still, including data-driven tests (parameterised tests), support dynamic generation of test data and boost the performance of running tests.

\section{For block-based programming languages}\label{sec:for-block-based-programming-languages}

TODO


\end{document}
